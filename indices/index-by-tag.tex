\documentclass{gwiki}
\usepackage{gwiki-links}

\Title{Master Index (by Tags)}
\Tags{meta, index}

\begin{document}
\NoteHeader

This is a comprehensive index of all notes in GWiki.

\section*{3-manifolds}

\begin{lst}
\item \wref{3-manifold} --- Orientable 3-manifolds can be built by Dehn surgery on framed links in $S^3$, so surgery data presents every example.
\end{lst}

\section*{TQFT}

\begin{lst}
\item \wref{(n+epsilon)D TQFT} --- Definition of $(n+\varepsilon)$-dimensional TQFTs via systems of topological fields and local relations, and the construction of skein modules and skein categories.
\end{lst}

\section*{algebra}

\begin{lst}
\item \wref{dg algebra} --- Differential graded algebras as monoids in chain complexes and their module theory.
\end{lst}

\section*{category-theory}

\begin{lst}
\item \wref{category} --- A category consists of objects and morphisms with composition satisfying associativity and identity laws.
\item \wref{functor} --- A functor transports objects and morphisms between categories while preserving identities and composition; it is the structure-preserving map between categories.
\item \wref{natural transformation} --- A natural transformation compares two functors by providing componentwise morphisms that commute with every arrow in the source category, so one functor transforms into the other uniformly.
\end{lst}

\section*{definition}

\begin{lst}
\item \wref{category} --- A category consists of objects and morphisms with composition satisfying associativity and identity laws.
\item \wref{functor} --- A functor transports objects and morphisms between categories while preserving identities and composition; it is the structure-preserving map between categories.
\item \wref{natural transformation} --- A natural transformation compares two functors by providing componentwise morphisms that commute with every arrow in the source category, so one functor transforms into the other uniformly.
\end{lst}

\section*{differential-geometry}

\begin{lst}
\item \wref{(pseudo-)Riemannian metric} --- Positive-semidefinite and positive-definite quadratic data on vector spaces and manifolds, including canonical models.
\item \wref{parallel transport via a connection} --- Explicit formula for parallel transport using a linear connection, deriving the path-ordered exponential and holonomy formula.
\end{lst}

\section*{functional-analysis}

\begin{lst}
\item \wref{Gelfand transform} --- The Gelfand transform embeds a commutative Banach algebra into continuous functions on its character space, exposing spectral data as evaluation.
\end{lst}

\section*{gauge-theory}

\begin{lst}
\item \wref{parallel transport via a connection} --- Explicit formula for parallel transport using a linear connection, deriving the path-ordered exponential and holonomy formula.
\end{lst}

\section*{higher-categories}

\begin{lst}
\item \wref{(n+epsilon)D TQFT} --- Definition of $(n+\varepsilon)$-dimensional TQFTs via systems of topological fields and local relations, and the construction of skein modules and skein categories.
\end{lst}

\section*{higher-category-theory}

\begin{lst}
\item \wref{cohomology}
\end{lst}

\section*{homotopy-theory}

\begin{lst}
\item \wref{cohomology}
\end{lst}

\section*{low-dimensional-topology}

\begin{lst}
\item \wref{3-manifold} --- Orientable 3-manifolds can be built by Dehn surgery on framed links in $S^3$, so surgery data presents every example.
\end{lst}

\section*{nPOV}

\begin{lst}
\item \wref{cohomology}
\end{lst}

\section*{operator-algebras}

\begin{lst}
\item \wref{Gelfand transform} --- The Gelfand transform embeds a commutative Banach algebra into continuous functions on its character space, exposing spectral data as evaluation.
\end{lst}

\section*{order-theory}

\begin{lst}
\item \wref{Birkhoff Duality} --- Finite distributive lattices are dual to finite posets via ideals and sets of join-irreducibles.
\end{lst}

\section*{physics}

\begin{lst}
\item \wref{parallel transport via a connection} --- Explicit formula for parallel transport using a linear connection, deriving the path-ordered exponential and holonomy formula.
\end{lst}

\section*{skein-modules}

\begin{lst}
\item \wref{(n+epsilon)D TQFT} --- Definition of $(n+\varepsilon)$-dimensional TQFTs via systems of topological fields and local relations, and the construction of skein modules and skein categories.
\end{lst}

\section*{Untagged}

\begin{lst}
\item \wref{abelian category}
\item \wref{Algebraic Bianchi Identity}
\item \wref{cochain complex}
\item \wref{Donaldson theory}
\item \wref{Drinfeld twist of a Hopf algebra}
\item \wref{Hstar monad}
\item \wref{Haag Duality}
\item \wref{looping and delooping}
\end{lst}


\end{document}
