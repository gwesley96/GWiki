\documentclass{gwiki}
\usepackage{gwiki-links}

\Title{Master Index (by Tags)}
\Tags{meta, index}

\begin{document}
\NoteHeader

This is a comprehensive index of all notes in GWiki.

\section*{3-manifolds}

\begin{lst}
\item \wref{3-manifold} --- Orientable 3-manifolds can be built by Dehn surgery on framed links in $S^3$, so surgery data presents every example.
\end{lst}

\section*{TQFT}

\begin{lst}
\item \wref{(n+epsilon)D TQFT} --- Definition of $(n+\varepsilon)$-dimensional TQFTs via systems of topological fields and local relations, and the construction of skein modules and skein categories.
\item \wref{gluing formula for inner products}
\end{lst}

\section*{TQFTs}

\begin{lst}
\item \wref{2-Hilbert spaces give Hstar-algebras}
\end{lst}

\section*{category-theory}

\begin{lst}
\item \wref{category} --- A category consists of objects and morphisms with composition satisfying associativity and identity laws.
\item \wref{functor} --- A functor transports objects and morphisms between categories while preserving identities and composition; it is the structure-preserving map between categories.
\item \wref{natural transformation} --- A natural transformation compares two functors by providing componentwise morphisms that commute with every arrow in the source category, so one functor transforms into the other uniformly.
\end{lst}

\section*{debug}

\begin{lst}
\item \wref{test_fixes}
\end{lst}

\section*{definition}

\begin{lst}
\item \wref{category} --- A category consists of objects and morphisms with composition satisfying associativity and identity laws.
\item \wref{functor} --- A functor transports objects and morphisms between categories while preserving identities and composition; it is the structure-preserving map between categories.
\item \wref{natural transformation} --- A natural transformation compares two functors by providing componentwise morphisms that commute with every arrow in the source category, so one functor transforms into the other uniformly.
\end{lst}

\section*{differential-geometry}

\begin{lst}
\item \wref{(pseudo-)Riemannian metric} --- Positive-semidefinite and positive-definite quadratic data on vector spaces and manifolds, including canonical models.
\item \wref{(pseudo-)Riemannian metric} --- Alias for (Pseudo-)Riemannian Metric
\item \wref{(pseudo-)Riemannian metric} --- Alias for (Pseudo-)Riemannian Metric
\item \wref{parallel transport via a connection} --- Explicit formula for parallel transport using a linear connection, deriving the path-ordered exponential and holonomy formula.
\item \wref{(pseudo-)Riemannian metric} --- Alias for (Pseudo-)Riemannian Metric
\item \wref{(pseudo-)Riemannian metric} --- Alias for (Pseudo-)Riemannian Metric
\item \wref{(pseudo-)Riemannian metric} --- Alias for (Pseudo-)Riemannian Metric
\item \wref{(pseudo-)Riemannian metric} --- Alias for (Pseudo-)Riemannian Metric
\item \wref{(pseudo-)Riemannian metric} --- Alias for (Pseudo-)Riemannian Metric
\item \wref{(pseudo-)Riemannian metric} --- Alias for (Pseudo-)Riemannian Metric
\end{lst}

\section*{disk-like}

\begin{lst}
\item \wref{system of topological fields}
\item \wref{unitary TQFTs introduction}
\end{lst}

\section*{disk-like framework}

\begin{lst}
\item \wref{(n+epsilon)D TQFT} --- Definition of $(n+\varepsilon)$-dimensional TQFTs via systems of topological fields and local relations, and the construction of skein modules and skein categories.
\end{lst}

\section*{framework}

\begin{lst}
\item \wref{system of topological fields}
\item \wref{unitary TQFTs introduction}
\end{lst}

\section*{functional-analysis}

\begin{lst}
\item \wref{Gelfand transform} --- The Gelfand transform embeds a commutative Banach algebra into continuous functions on its character space, exposing spectral data as evaluation.
\end{lst}

\section*{gauge-theory}

\begin{lst}
\item \wref{parallel transport via a connection} --- Explicit formula for parallel transport using a linear connection, deriving the path-ordered exponential and holonomy formula.
\end{lst}

\section*{gluing}

\begin{lst}
\item \wref{gluing formula for inner products}
\end{lst}

\section*{higher-categories}

\begin{lst}
\item \wref{(n+epsilon)D TQFT} --- Definition of $(n+\varepsilon)$-dimensional TQFTs via systems of topological fields and local relations, and the construction of skein modules and skein categories.
\end{lst}

\section*{higher-category-theory}

\begin{lst}
\item \wref{cohomology}
\end{lst}

\section*{homotopy-theory}

\begin{lst}
\item \wref{cohomology}
\end{lst}

\section*{low-dimensional-topology}

\begin{lst}
\item \wref{3-manifold} --- Orientable 3-manifolds can be built by Dehn surgery on framed links in $S^3$, so surgery data presents every example.
\end{lst}

\section*{nPOV}

\begin{lst}
\item \wref{cohomology}
\end{lst}

\section*{operator-algebras}

\begin{lst}
\item \wref{Gelfand transform} --- The Gelfand transform embeds a commutative Banach algebra into continuous functions on its character space, exposing spectral data as evaluation.
\end{lst}

\section*{order-theory}

\begin{lst}
\item \wref{Birkhoff Duality} --- Finite distributive lattices are dual to finite posets via ideals and sets of join-irreducibles.
\end{lst}

\section*{physics}

\begin{lst}
\item \wref{parallel transport via a connection} --- Explicit formula for parallel transport using a linear connection, deriving the path-ordered exponential and holonomy formula.
\end{lst}

\section*{project}

\begin{lst}
\item \wref{2-Hilbert spaces give Hstar-algebras}
\end{lst}

\section*{skein-modules}

\begin{lst}
\item \wref{(n+epsilon)D TQFT} --- Definition of $(n+\varepsilon)$-dimensional TQFTs via systems of topological fields and local relations, and the construction of skein modules and skein categories.
\end{lst}

\section*{stub}

\begin{lst}
\item \wref{fermion}
\item \wref{H}
\item \wref{Levin–Wen Theory}
\end{lst}

\section*{talk}

\begin{lst}
\item \wref{Codimension-1 defects, categorified group actions, and condensing fermions}
\end{lst}

\section*{test}

\begin{lst}
\item \wref{test_fixes}
\item \wref{tikz test 01 patterns}
\item \wref{tikz test 02 markers}
\item \wref{tikz test 03 colors}
\item \wref{tikz test 04 arrows}
\item \wref{tikz test 05 decorations}
\item \wref{tikz test 06 string diagrams}
\item \wref{tikz test 07 multiline}
\item \wref{tikz test 08 knots}
\item \wref{tikz test 09 positioning}
\item \wref{tikz test 10 inline diagrams}
\item \wref{tikz test 11 layers}
\item \wref{tikz test 12 double strands}
\item \wref{tikz test 13 monoidal categories}
\item \wref{tikz test 14 shapes}
\item \wref{tikz test 15 compositions}
\item \wref{tikz test 16 3d effects}
\item \wref{tikz test 17 intersections}
\item \wref{tikz test 18 matrix}
\item \wref{tikz test 19 full example serpent}
\item \wref{tikz test 20 full example cobordism}
\end{lst}

\section*{tikz}

\begin{lst}
\item \wref{tikz test 01 patterns}
\item \wref{tikz test 02 markers}
\item \wref{tikz test 03 colors}
\item \wref{tikz test 04 arrows}
\item \wref{tikz test 05 decorations}
\item \wref{tikz test 06 string diagrams}
\item \wref{tikz test 07 multiline}
\item \wref{tikz test 08 knots}
\item \wref{tikz test 09 positioning}
\item \wref{tikz test 10 inline diagrams}
\item \wref{tikz test 11 layers}
\item \wref{tikz test 12 double strands}
\item \wref{tikz test 13 monoidal categories}
\item \wref{tikz test 14 shapes}
\item \wref{tikz test 15 compositions}
\item \wref{tikz test 16 3d effects}
\item \wref{tikz test 17 intersections}
\item \wref{tikz test 18 matrix}
\item \wref{tikz test 19 full example serpent}
\item \wref{tikz test 20 full example cobordism}
\end{lst}

\section*{unitary}

\begin{lst}
\item \wref{2-Hilbert spaces give Hstar-algebras}
\end{lst}

\section*{unitary-TQFT}

\begin{lst}
\item \wref{gluing formula for inner products}
\end{lst}

\section*{visualization}

\begin{lst}
\item \wref{tikz test 01 patterns}
\item \wref{tikz test 02 markers}
\item \wref{tikz test 03 colors}
\item \wref{tikz test 04 arrows}
\item \wref{tikz test 05 decorations}
\item \wref{tikz test 06 string diagrams}
\item \wref{tikz test 07 multiline}
\item \wref{tikz test 08 knots}
\item \wref{tikz test 09 positioning}
\item \wref{tikz test 10 inline diagrams}
\item \wref{tikz test 11 layers}
\item \wref{tikz test 12 double strands}
\item \wref{tikz test 13 monoidal categories}
\item \wref{tikz test 14 shapes}
\item \wref{tikz test 15 compositions}
\item \wref{tikz test 16 3d effects}
\item \wref{tikz test 17 intersections}
\item \wref{tikz test 18 matrix}
\item \wref{tikz test 19 full example serpent}
\item \wref{tikz test 20 full example cobordism}
\end{lst}

\section*{Untagged}

\begin{lst}
\item \wref{(∞,1)-category}
\item \wref{(∞,1)-pullback in an (∞,1)-category}
\item \wref{2-condensation in a 2-category}
\item \wref{abelian category}
\item \wref{Algebraic Bianchi Identity}
\item \wref{Banach algebra}
\item \wref{Bernoulli scheme, Markov chain}
\item \wref{bounded spread Haag duality}
\item \wref{Calculus on Riemannian manifolds}
\item \wref{categorified group action}
\item \wref{Cauchy complete n-category}
\item \wref{cochain complex}
\item \wref{components in presemisimple linear 2-categories}
\item \wref{defects of a QFT}
\item \wref{Dehn surgery}
\item \wref{DG-category}
\item \wref{disk-like n-category}
\item \wref{Donaldson theory}
\item \wref{double of a manifold}
\item \wref{Drinfeld twist of a Hopf algebra}
\item \wref{Eckmann–Hilton Argument}
\item \wref{Eilenberg–MacLane space}
\item \wref{enriched category}
\item \wref{equivalence in an (∞,n)-category}
\item \wref{Existence and Uniqueness Theorem for ODEs}
\item \wref{extended isotopy}
\item \wref{exterior derivative}
\item \wref{Freed's 4-3-2 8-7-6}
\item \wref{Gelfand–Kirillov dimension}
\item \wref{H*-bimodule}
\item \wref{Hstar monad}
\item \wref{Haag Duality}
\item \wref{Hopf algebra}
\item \wref{Jacobi Identity}
\item \wref{knot}
\item \wref{linear connection 1-form}
\item \wref{linear connection in a vector bundle}
\item \wref{local relations for a system of topological fields}
\item \wref{looping and delooping}
\item \wref{mapping class group}
\item \wref{modular fusion category}
\item \wref{monoid in a monoidal category}
\item \wref{n-condensation in an n-category}
\item \wref{orientable manifold}
\item \wref{pointed object}
\item \wref{positive-definite form}
\item \wref{Postnikov tower}
\item \wref{premodular fusion category}
\item \wref{presemisimple 2-category}
\item \wref{quadratic form}
\item \wref{ribbon category}
\item \wref{Riemann (0,4)-Curvature Tensor}
\item \wref{ring (nPOV)}
\item \wref{signature of a manifold}
\item \wref{smooth manifold}
\item \wref{spherical category}
\item \wref{Tensor Field}
\item \wref{terminal object in an (∞,1)-category}
\item \wref{test_spacing}
\item \wref{twist}
\end{lst}


\end{document}
