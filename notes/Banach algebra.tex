\documentclass{gwiki}
\usepackage{gwiki-links}

\Title{Banach algebra}
\Tags{}

\begin{document}
\NoteHeader

A \wref[Banach algebra]{ShortC*AlgCourse.pdf#page=9&selection=88,0,88,14&color=important} is a \wref{normed algebra} that is a \wref{Banach space}, i.e., that is \wref[complete]{metric space} with respect to its \wref[norm]{normed vector space}.

Notice that a subalgebra is itself an algebra. A subalgebra of a normed algebra is a normed algebra. The closure of a subalgebra of a normed algebra is a normed algebra. Therefore the closure of any subalgebra of a Banach algebra is again a Banach algebra.

\subsubsection*{Examples}
\textbf{\wref[Proposition 1.9]{ShortC*AlgCourse.pdf#page=11&selection=224,0,224,15&color=yellow}.} If $\mathcal{X}$ is a Banach space, then $B(\mathcal{X})$ is a unital Banach algebra.

\textbf{\wref[Example 1.3]{ShortC*AlgCourse.pdf#page=9&selection=93,0,95,3&color=yellow}.}


See also:
- \wref{normed vector space}

\end{document}
