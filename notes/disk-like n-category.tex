\documentclass{gwiki}
% noconvert
\Title{disk-like n-category}
\Tags{}
\newcommand{\gwikicreateddate}{October 9, 2025 at 12:24 pm ET}

\begin{document}

\NoteHeader
\NoteNavigation

Let $H$ be a given geometric structure (e.g., none, orientation, spin, pin, $\dots$). All manifolds are assumed to be PL. All $k$-manifolds are assumed equipped with the germ of a thickening to an $n$-manifold and a choice of $H$-structure and an $n$-framing in the thickening. By “homeomorphism” we mean $H$-structure-preserving PL homeomorphism of the (germ of the thickened) manifold. By a “$k$-ball” or “$k$-disk” we mean a $k$-manifold that is homeomorphic to the closed $k$-ball $D^n$ in $\bbR^{k}$ via some (unspecified!) homeomorphism. Denote by $\overline{\,\cdot\,}$ the \defn{$H$-structure-reversal} involutive endofunctor on the groupoid $\{k\text{-balls and homeomorphisms}\}$.

\begin{definition}
An (unenriched, $H$-pivotal) \defn{disk-like $n$-category} $\cC$ consists of the following data.
\begin{lst}
  \item For each $0\leq k \leq n$, a functor $\cC^{k}\colon \{k\text{-balls and homeomorphisms}\}\to \mathsf{Set}$.
  \item For each $k\leq n$, a \defn{conjugation} natural isomorphism $\widehat{\;\cdot\;}\colon \cC^{k}(\overline{\,\cdot\,})\cong \cC^{k}$.
  \item For each $(k-1)$-ball $Y$ in the boundary of a $k$-ball $X$, a \defn{restriction map} $\cC^{k}(X)\to \cC^{k-1}(Y)$ (defined on sufficiently large subsets of $\cC^{k}$).
  \item For each $k$-ball $X$ obtained by gluing two $k$-balls $X_{1}$ and $X_{2}$ along a common $(k-1)$-ball $Y$ in their boundaries, a \defn{composition/gluing map} $\cC^{k}(X_{1})\times_{\cC^{k-1}(Y)}\cC^{k}(X_2)\to \cC^{k}(X)$.
  \item For each $k$-ball $X$ and $m$-ball $D$ (with $k+m\leq n$), a \defn{product map} $\cC^{k}(X)\to \cC^{k+m}(X\times D)$ taking a morphism $a$ to the \defn{product/identity morphism} $a\times D$.
\end{lst}
These data are subject to the following conditions.
\begin{lst}
  \item The gluing maps are strictly associative and are compatible (commute) with the actions of diffeomorphisms.
  \item The product maps are associative and are compatible with gluing, restrictions, and diffeomorphisms.
  \item Diffeomorphisms of $n$-balls that are isotopic rel boundary act identically; the same holds for \wref[collar maps]{system of topological fields}.
  \item Morphisms are \defn{splittable}: for any $k$-morphism $c$ (with $k<n$), there exists an embedded cell complex $S_c$ (the \defn{string locus}) within its underlying ball, such that $c$ is splittable along any decomposition that is transverse to $S_c$.
\end{lst}
\end{definition}

\begin{example}[$\cC$-labeled string diagrams]
Our main example of an $n$-dimensional system of fields will be $\cC$-labeled string diagrams for some “traditional $n$-category with strong duality” $\cC$. Recall that a $\cC$-\defn{labeled string diagram} on a $k$-manifold $Y$ is a general-position embedded cell complex of $Y$ whose codimension-$j$ cells are labeled by $j$-morphisms of $\cC$.
\end{example}

\SeeAlso{gluing formula for inner products}

\References

\Footer

\end{document}
