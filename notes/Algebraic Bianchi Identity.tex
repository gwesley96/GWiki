\documentclass{gwiki}

\Title{Algebraic Bianchi Identity}
\Tags{}

\begin{document}

\NoteHeader

The \textbf{algebraic Bianchi identity} (AKA \textbf{first Bianchi identity}) is the assertion that the first three slots of the \wref{riemann-(04)-curvature-tensor} satisfy the \wref{jacobi-identity}, i.e., that$$\mathrm{Rm}(u,v,w,q)+\mathrm{Rm}(w,u,v,q)+\mathrm{Rm}(v,w,u,q)=0\qquad\forall v,w,u\in\mathfrak{X}(M),$$or in local coordinates, $R_{ijkl}+R_{jkil}+R_{kijl}=0$ for all $i,j,k$.

This is because we can view $R(-,-)$ as a section of $\Omega^{2}(M)\otimes \mathrm{End}(E)$ with coefficients in the Lie algebra $\mathfrak{gl}(\mathrm{O}(E_{x}))$; see ((Calegari, p. 53)).

See also: 
\begin{itemize}
\item \wref{riemann-(04)-curvature-tensor}
\item \wref{riemann-(13)-curvature-tensor}
\item \wref{differential-bianchi-identity}
\end{itemize}

\Footer

\end{document}

