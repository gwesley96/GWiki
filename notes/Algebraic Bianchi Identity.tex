\documentclass{gwiki}

\Title{Algebraic Bianchi Identity}
\Tags{}
\Summary{The first Bianchi identity symmetrically sums the curvature tensor to zero over its first three slots.}

\begin{document}

\KeySources{\nlab{Bianchi identity}}

\NoteHeader

\begin{framedidea}[Core idea]
Viewing curvature as a $\mathfrak{gl}$-valued $2$-form forces a Jacobi-type symmetry: cycling the first three inputs of the Riemann tensor gives zero.
\end{framedidea}

The \textbf{algebraic Bianchi identity} (AKA \textbf{first Bianchi identity}) is the assertion that the first three slots of the \wref{Riemann (0,4)-Curvature Tensor} satisfy the \wref{Jacobi Identity}, i.e., that$$\mathrm{Rm}(u,v,w,q)+\mathrm{Rm}(w,u,v,q)+\mathrm{Rm}(v,w,u,q)=0\qquad\forall v,w,u\in\mathfrak{X}(M),$$or in local coordinates, $R_{ijkl}+R_{jkil}+R_{kijl}=0$ for all $i,j,k$.

This is because we can view $R(-,-)$ as a section of $\Omega^{2}(M)\otimes \mathrm{End}(E)$ with coefficients in the Lie algebra $\mathfrak{gl}(\mathrm{O}(E_{x}))$; see ((Calegari, p. 53)).

\SeeAlso{Riemann (0,4)-Curvature Tensor,riemann-(13)-curvature-tensor,differential-bianchi-identity}

\References

\Footer

\end{document}
