\documentclass{gwiki}

\Title{Cohomology}
\Tags{higher-category-theory, homotopy-theory, nPOV}

\begin{document}

\NoteHeader

The nPOV (homotopy-theoretic) definition of cohomology in an ambient $(\infty,1)$-category, emphasizing cocycles, coboundaries, and cohomology classes.

Let $\mathcal{C}$ be an ambient $(\infty,1)$-category (most often an $(\infty,1)$-topos or a stable $(\infty,1)$-category).

Fix objects $X$ (the **domain object**) and $A$ (the **coefficient object**) in $\mathcal{C}$.

\begin{definition}[Cohomology]\label{def:cohomology}

  - The $A$**-cocycle space** on $X$ is the space $\mathcal{C}(X\to A)$, often denoted $\mathrm{Z}(X;A)$.
  - An $A$**-cocycle** on $X$ is an element $\omega\in \mathcal{C}(X\to A)$.
  - The **$A$-cohomology set** of $X$ (AKA the **cohomology** of $X$ with $A$**-coefficients**) is the *set* of connected components of the $A$-cocycle space
  \[
    \mathrm{H}(X;A)\coloneqq\pi_{0}\mathcal{C}(X\to A).
  \]
  - The **cohomology class** of an $A$-cocycle $\omega$ is its component $[\omega]\in\mathrm{H}(X;A)$.
  
    - If $\mathcal{C}$ has a terminal object $*$ and $A$ is equipped with a pointing $a_0\colon *\to A$, then the **trivial $A$-cocycle** $\omega_{\mathrm{triv}}$ on $X$ is the $A$-cocycle $X\to *\mathbin{\smash{\overset{a_0}{\to}}} A\in\mathcal{C}(X\to A)$.
  
  - An $A$**-coboundary** is a 2-morphism $\eta\in \mathcal{C}(\omega_{1}\Rightarrow\omega_{2})$.
  - Two cocycles $\omega_{1},\omega_{2}$ on $X$ are $A$**-cohomologous** if they are connected by an $A$-coboundary, or equivalently if $[\omega_{1}]=[\omega_{2}]$ in $\mathrm{H}(X;A)$.
  
    - If $A$ is equipped with a pointing $a_0\colon *\to A$, then a cocycle $\omega$ is an $A$-**coboundary** if it is cohomologous to the trivial cocycle $\omega_{\mathrm{triv}}$.
  
  - For an object $K\in \mathcal{C}$, an $A$-cocycle $\mathscr{c}$ on $K$, and a $K$-cocycle $\omega$ on $X$, the **characteristic class of $\omega$ with respect to $\mathscr{c}$** is the $A$-cohomology class of the composite cocycle
  \[
    \mathscr{c}(\omega)\coloneqq [\mathscr{c}\circ \omega]\in\mathrm{H}(X;A).
  \]
  - For a $K$-cocycle $g\colon X\to K$, one says that its homotopy fiber $P\to X$ (meaning $(\infty,1)$-pullback of the cospan $X\mathbin{\smash{\overset{g}{\to}}}K\leftarrow *$), is the object **classified by the $K$-cohomology class** $[g]\in\mathrm{H}(X;K)$.
  
    - When $\mathcal{C}$ is an $(\infty,1)$-topos and $K$ is pointed, such an object usually has the interpretation of a principal $\infty$-bundle (e.g., principal bundles, gerbes, principal 2-bundles, and so on).
    - If the domain object $X$ itself is a group object, then $P\to X$ is a group extension, and in this case we often write $\mathrm{Ext}(X;A)$ instead of $\mathrm{H}(X;A)$, and refer to elements of $\mathrm{Ext}(X;A)$ as extensions instead of cocycles.
  

\end{definition}


## Examples


\begin{example}[Degree-$n$ cohomology]
If $A$ admits an $n$-fold \wref[delooping]{looping and delooping} $\mathrm{B}^{n}A$, then the **degree-$n$ $A$-cohomology of $X$** is the $\mathrm{B}^{n}\kern-.25exA$-cohomology set
\[
  \mathrm{H}^{n}(X;A)\coloneqq\mathrm{H}(X;\mathrm{B}^{n}\kern-.25exA).
\]
\end{example}

\begin{example}[Cochain complex cohomology]
If $C^\bullet$ is a \wref{cochain complex} in an \wref{abelian category} $\mathcal{A}$, then the $q$th **cohomology group** is defined by
\[
  \mathrm{H}^q(C^\bullet)\coloneqq \mathrm{ker}(C^q\overset{\delta}{\to} C^{q+1})/\mathrm{im}(C^{q-1}\overset{\delta}{\to} C^q).
\]
The numerator consists of **$q$-cocycles**, and the denominator consists of **$q$-coboundaries**.
\end{example}

\SeeAlso{twisted cohomology, de Rham cohomology, singular (co)homology, abelian cohomology, nonabelian cohomology, differential cohomology, characteristic class}

\References

\Footer

\end{document}
