\documentclass{gwiki}

\Title{Birkhoff Duality}
\Tags{order-theory}{lattice-theory}
\Summary{Finite distributive lattices are dual to finite posets via ideals and sets of join-irreducibles.}

\begin{document}


\NoteHeader

\begin{idea}[Idea]
Birkhoff duality identifies a finite distributive lattice with the poset of its join-irreducible elements, while the lattice itself is recovered as the lattice of order ideals of that poset. Morphisms become order-preserving maps in the opposite direction.
\end{idea}

\KeySources{\nlab{Birkhoff duality}}

\begin{definition}[Duality statement]
Let $\mathsf{FDLat}$ be the category of finite distributive lattices and $\mathsf{FPos}$ the category of finite posets. The functor sending $L$ to its poset of join-irreducibles $J(L)$ and $P$ to its lattice of ideals $\mathrm{Idl}(P)$ yields an equivalence $\mathsf{FDLat} \simeq \mathsf{FPos}^{\mathrm{op}}$.
\end{definition}

\SeeAlso{category,functor,natural transformation}
\IncomingLinks{dg algebra}

\References

\Footer

\end{document}
