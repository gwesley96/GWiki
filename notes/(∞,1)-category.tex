\documentclass{gwiki}
\usepackage{gwiki-links}

\Title{(∞,1)-category}
\Tags{stub}

\begin{document}
\NoteHeader

\begin{definition}[\href{../pdfs/LurieHTT.pdf#page=32&annotation=13175R}{HTT.1.1.2.4}]
An \defn{$(\infty,1)$-category} (AKA \defn{$\infty$-category}) is a \wref{simplicial set} $\mathcal{C}$ in which all \wref[inner horns]{inner horn} have \wref[fillers]{filler}.
Every (ordinary) \wref{category} $\mathcal{C}$ can be viewed as an $(\infty,1)$-category by taking the \wref[nerve]{nerve of a category} $\mathrm{N}\mathcal{C}$.
\end{definition}

\begin{remark}[\href{../pdfs/Camarena∞Whirlwind.pdf#page=2&annotation=296R}{Remark}]
Many definitions and statements of results from ordinary category theory generalize straightforwardly to $(\infty, 1)$-categories, often simply by replacing bijections of Hom-sets with weak homotopy equivalences of mapping spaces, but with current technology the traditional proofs do not generalize, and instead often require delicate model specific arguments: most of this work has been done using the model of quasicategories.
\end{remark}

\begin{seealso}
\item \wref{models of (∞,1)-categories}
\item \wref{∞-category of spaces}
\item \wref{∞-category of spectra}
\item \wref{stable (∞,1)-category}
\item \href{https://emilyriehl.github.io/files/elements.pdf}{Elements of ∞-category theory} (Riehl–Verity)
\item see the diagrams of \href{https://ncatlab.org/nlab/show/Lie+n-groupoid}{this page}
\item \href{../pdfs/icats.pdf}{icats.pdf} (a fantastic set of notes)
\end{seealso}

\end{document}
