\documentclass{gwiki}

\Title{Category}
\Tags{category-theory, definition}
\Summary{A category consists of objects and morphisms with composition satisfying associativity and identity laws.}

\begin{document}

\NoteHeader

\begin{framedidea}[Idea]
Categories abstract the common structure shared by mathematical contexts: objects (like sets, groups, topological spaces) and structure-preserving maps between them (functions, homomorphisms, continuous maps), formalized through composition and identities.
\end{framedidea}

\begin{definition}[Category]\label{def:category}
A \textbf{category} $\cC$ consists of:
\begin{lst}
  \item A class $\Obj(\cC)$ of \emph{objects}
  \item For each pair $A, B$, a set $\Hom(A, B)$ of \emph{morphisms}
  \item An identity $\id_A : A \to A$ for each object
  \item Composition $\circ : \Hom(B,C) \times \Hom(A,B) \to \Hom(A,C)$
\end{lst}
satisfying associativity and unit laws.
\end{definition}

\section{Examples}

\begin{example}
Standard categories:
\begin{lst}
  \item $\Set$: sets and functions
  \item $\Grp$: groups and homomorphisms
  \item $\Top$: topological spaces and continuous maps
  \item $\Vec_k$: vector spaces and linear maps
\end{lst}
\end{example}

\seealso{functor, natural transformation}

\References

\Footer

\end{document}
