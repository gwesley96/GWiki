\documentclass{gwiki}

\Title{Seiberg–Witten Theory}
\Tags{}
\newcommand{\gwikicreateddate}{November 30, 2025 at 11:43 pm ET}

\begin{document}

\NoteHeader
\NoteNavigation

Pure SU(2) super Yang-Mills theory This is a 4d N = 2 theory, which admits a topological twist giving rise to a Witten-type TQFT. At high energy (i.e., small distance scales), the topological theory is called Donaldson theory, and is described by the ASD equations. At low energy (high distance scales), it gives rise to SeibergWitten theory, described by the Seiberg-Witten equations. \wcite{Freed+QFT\&ManifoldInvariants.pdf#page=14\&selection=80,0,106,44}

See also:
\begin{lst}
  \item \wref{Donaldson Theory}
\end{lst}

\References

\Footer

\end{document}
