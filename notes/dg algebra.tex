\documentclass{gwiki}

\Title{DG-algebra}
\Tags{algebra}{homological-algebra}
\Aliases{DGA, differential graded algebra}
\Summary{Differential graded algebras as monoids in chain complexes and their module theory.}

\begin{document}

\NoteHeader

\begin{idea}[Idea]
A differential graded algebra packages a chain complex with a compatible multiplication, so it behaves like an algebra object in the monoidal category of chain complexes; modules over it are enriched functors out of the associated one-object DG-category.
\end{idea}

A \textbf{DG-algebra} (AKA \textbf{DGA}) is a monoid $A$ in $(\mathrm{Ch}_{\mathbb{k}},\otimes _{\mathbb{k}})$ viewed as a one-object enriched category $\underline{A}$ (without DG-structure).

A \textbf{DG-module} $\mathcal{M}$ over a DGA $A$ is a functor $\underline{A}\to \mathrm{Ch}_{k}$ (where $\mathrm{Ch}_{k}$ is enriched).

For a quick primer on DGAs, see \arxiv{0601185}.

\SeeAlso{category,functor,natural transformation}
\IncomingLinks{(pseudo-)Riemannian metric}

\References

\Footer

\end{document}
