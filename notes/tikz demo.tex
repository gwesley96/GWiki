\documentclass{gwiki}

\Title{TikZ string diagrams demo}
\Tags{tikz, visualization, category theory}
\Aliases{TikZ demo, string diagram examples}
\Topics{String diagrams, coupons, patterns, arrows, color schemes, quick reference examples.}

\begin{document}

\NoteHeader

\begin{idea}
The \texttt{tz.sty} package provides convenient macros for drawing string diagrams, commutative diagrams, and other mathematical visualizations using TikZ, with support for layers, patterns, and specialized node styles.
\end{idea}

\section{Basic String Diagrams}

The \texttt{tz.sty} package provides inline TikZ via the \verb|\tz{}| command and the \texttt{tkz} environment. Here's a simple example:

\begin{center}
\begin{tkz}
  \node at (0,0) {See \wref{functor} and \wref[category theory]{category}};
\end{tkz}
\end{center}

\begin{example}[Inline diagram]
A morphism can be drawn inline: \tz{\draw[thick] (0,0) -- (1,0); \fill (0,0) circle (1.5pt); \fill (1,0) circle (1.5pt);}.
\end{example}

\section{Coupons and String Diagrams}

String diagrams use coupons (boxes) connected by strands:

\begin{center}
\begin{tkz}
  % Draw strands
  \draw[thick] (0,0) -- (0,1.5);
  \draw[thick] (2,0) -- (2,1.5);

  % Draw coupon
  \cpn{(1,0.75)}{$f$}

  % Connect to coupon
  \draw[thick] (0,1.5) -- (0.6,0.75);
  \draw[thick] (2,1.5) -- (1.4,0.75);

  % Label endpoints
  \node[below] at (0,0) {$A$};
  \node[below] at (2,0) {$B$};
\end{tkz}
\end{center}

\section{Patterns and Regions}

The package includes several pattern styles for indicating different types of regions:

\begin{center}
\begin{tkz}
  \fill[primedregion=blue!20] (0,0) rectangle (1.5,1.5);
  \fill[boxregion=red!20] (2,0) rectangle (3.5,1.5);
  \fill[diagregion=green!20] (4,0) rectangle (5.5,1.5);

  \node at (0.75,0.75) {Dots};
  \node at (2.75,0.75) {Boxes};
  \node at (4.75,0.75) {Lines};
\end{tkz}
\end{center}

\section{Arrow Styles}

Various arrow decorations are available:

\begin{center}
\begin{tkz}
  \draw[thick,mid>] (0,0) -- (2,0) node[right] {Mid arrow};
  \draw[thick,arrow at=0.3] (0,-.5) -- (2,-.5) node[right] {Arrow at 30 percent};
  \draw[thick,Rightarrow] (0,-1) -- (2,-1) node[right] {Double arrow};
\end{tkz}
\end{center}

\section{Knot Diagrams}

Drawing overcrossings and undercrossings:

\begin{center}
\begin{tkz}
  % First strand (goes over)

  % Second strand (goes under)
  \draw[thick] (1,-0.5) -- (1,0.5);
  \draw[thick,over] (0,0) .. controls (0.5,0.5) and (1.5,0.5) .. (2,0);

  \node[below] at (1,-0.5) {Crossing};
\end{tkz}
\end{center}

\section{Color Schemes}

Predefined color scheme for consistency:

\begin{center}
\begin{tkz}
  \foreach \i/\col in {0/colA,1/colB,2/colC,3/colD,4/colE} {
    \fill[\col] (\i*1.2,0) circle (0.4);
    \node[below] at (\i*1.2,-0.5) {\texttt{\col}};
  }
\end{tkz}
\end{center}

\section{Conclusion}

The \texttt{tz.sty} package integrates seamlessly with GWiki's document class, providing powerful tools for mathematical visualization while maintaining clean, readable source code.

\SeeAlso{category,functor}

\References

\Footer

\end{document}
