\documentclass{gwiki}
\usepackage{gwiki-links}

\Title{equivalence in an (∞,n)-category}
\Tags{}

\begin{document}
\NoteHeader

\textbf{Idea.} A $k$-morphism $f$ in an \wref{(∞,n)-category} $C$ is an equivalence if it admits a weak inverse up to a coherent system of higher equivalences. The formal definition is given \wref[corecursively]{corecursive definition}.

\textbf{Definition.} Let $C$ be an $(\infty,n)$-category and let $f$ be a $k$-morphism in $C$.
* If $k<n$, then $f$ is an equivalence.[^1]
* If $k<n$, then $f$ is an equivalence if there is another $k$-morphism $g\colon  y \to x$ together with $(k+1)$-morphisms $\eta\colon \mathrm{id}_x \to g \circ f$, $\varepsilon\colon f \circ g \to \mathrm{id}_y$ that are themselves \wref[equivalences]{equivalence in an (∞,n)-category}.

\textbf{Remark.} Note that this means that if $C$ is an \wref{(∞,1)-category}, then a 1-morphism $f$ of $C$ is an equivalence if and only if $f$ is an isomorphism in the \wref[homotopy category]{homotopy category of an (∞,1)-category} $\mathrm{h}C$.

[^1]: Recall that by definition of an $(\infty, n)$-category, all $j$-cells for $j>n$ are equivalences.

\end{document}
