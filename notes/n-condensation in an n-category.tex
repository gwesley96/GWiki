\documentclass{gwiki}
\usepackage{gwiki-links}

\Title{n-condensation in an n-category}
\Tags{}

\begin{document}
\NoteHeader

Condensations categorify \wref[split surjections]{split epimorphism}.

\textbf{\wref[Definition 2.1.1]{GJ19CondensationsInHigherCategories.pdf#page=10&selection=4,0,4,15}.} A \textbf{0-condensation} is an equality of objects in a 0-category (a set). Suppose that $a$ and $b$ are objects of a weak $n$-category $\mathcal{C}$. An \textbf{$n$-condensation} of $a$ onto $b$, written $\,\longrightarrow\!\!\!\!\!\!\!\!\!\hookleftarrow{\!\!\!\!\!\!\!\!\!\!\color{white}\blacksquare\color{black}}\,\,b\!\!\!\!\!\!\!\!\!\!\!\!\!\!\!a\,\,\,\,\,\,\,\,\,\,\,\,$, is a pair of 1-morphisms ${}_aX_b\in\mathcal{C}(a\to b)$, ${}_bY_a\in\mathcal{C}(b\to a)$ together with an $(n-1)$-condensation $\,\,\,\,\,\,\,\,\,\,\,\,\,\,\,\,\,\,\,\,\,\,\,\longrightarrow\!\!\!\!\!\!\!\!\!\hookleftarrow{\!\!\!\!\!\!\!\!\!\!\color{white}\blacksquare\color{black}}\,\,1_{b}\!\!\!\!\!\!\!\!\!\!\!\!\!\!\!\!\!\!\!\!\!\!\!\!\!\!\!\!\!\!\!\!\!\!\!\!\!\!\!{}_bY\otimes_a X_{b}\,\,\,\,\,\,\,\,\,\,\,\,\,\,\,$.
```tikz
\usepackage{amsmath,amssymb}
\usetikzlibrary{decorations.markings,arrows,decorations.pathreplacing,arrows,decorations.pathmorphing,decorations.pathreplacing,decorations.markings,shapes.geometric,through,fit,shapes.symbols}
\begin{document}
\tikzset{every picture/.style={scale=2,very thick,yscale=0.8}}
\tikzset{condensationlarge/.style={-, decoration={markings, mark=at position -1ex with {\arrow[scale=1.25,very thick]{left hook}}, mark=at position 1 with {\arrow[scale=1.75]{>}}, }, preaction={decorate}}}
\tikzset{doublecondensationlarge/.style={double, double distance=1.5pt, shorten <=6pt, shorten >=6pt, decoration={markings, mark=at position -15pt with {\arrow[scale=1.5,thick]{left hook}}, mark=at position -4.25pt with {\arrow[scale=2,thick]{>}}, }, preaction={decorate}}}

\Huge
\begin{tikzpicture}[baseline=(midpoint)]

\path (0,0) node (X) {$a$} (0,-2) node (Y) {$b$} (0,-1) coordinate (midpoint);

\draw[condensationlarge] (X) -- (Y);
\end{tikzpicture}

\quad = \!\!\quad
\Huge
\begin{tikzpicture}[baseline=(midpoint)]

\path (0,0) node (X) {$a$} (0,-.35) coordinate (Xe) (0,-2) node (Y) {$b$} (0,-1) coordinate (midpoint);

\draw[->] (X) .. controls +(-.75,-.75) and +(-.75,.75) .. node[auto,swap,font=\Large] {${}_aX_b$} (Y);


\draw[->] (Y) .. controls +(.75,.75) and +(.75,-.75) .. node[auto,swap,font=\Large] {${}_bY_a$} (X);

\path (Y.north) +(.1,0) coordinate (Ytopa) +(-.1,0) coordinate (Ytopb);

\draw[->] (Ytopa) .. controls +(.6,.6) and +(.5,0) .. (Xe) .. controls +(-.5,0) and +(-.6,.6) .. (Ytopb);

\draw[doublecondensationlarge] (Xe) -- (Y);

\end{tikzpicture}
\end{document}
```
We say that $a$ (resp. $X$) \textbf{condenses onto} $b$.

\textbf{\wref[Example 2.1.2]{GJ19CondensationsInHigherCategories.pdf#page=10&selection=121,0,121,13}.} In a 1-category $\mathcal{C}$, to say an object $a$ condenses onto an object $b$ means that there is a split surjection (AKA retract). Indeed, the 0-condensation of $X\circ Y$ onto $\mathrm{id}_{b}$ is an equality, so $X\colon a\twoheadrightarrow b$ is an epimorphism ("surjection") and $a\hookleftarrow b\!:Y$ is a monomorphism ("split"). This is the reason for the notation.

For a 2-category $\mathfrak{C}$, to say an object $a$ condenses onto an object $b$ means that there are 1-morphisms ${}_aX_b,{}_{b}Y_{a}\in \mathrm{End}(c)$ and a 1-condensation $\,\,\,\,\,\,\,\,\,\,\,\,\,\,\,\,\,\,\,\,\,\,\,\longrightarrow\!\!\!\!\!\!\!\!\!\hookleftarrow{\!\!\!\!\!\!\!\!\!\!\color{white}\blacksquare\color{black}}\,\,1_{b}\!\!\!\!\!\!\!\!\!\!\!\!\!\!\!\!\!\!\!\!\!\!\!\!\!\!\!\!\!\!\!\!\!\!\!\!\!\!\!{}_bY\otimes_a X_{b}\,\,\,\,\,\,\,\,\,\,\,\,\,\,\,$, i.e., maps $\eta\in\mathfrak{C}(1_b\Rightarrow  {}_{b}Y\otimes_{a}X_{b})$ and $\varepsilon\in\mathfrak{C}({}_{b}Y\otimes_{a} X_{b}\Rightarrow 1_{b})$ with $\varepsilon\eta=\mathrm{id}_{1_{b}}$.

!\wref{2-condensation in a 2-category}

See also: 
- \wref{condensation algebra in a monoidal (n-1)-category}
- \wref{condensation monad in an n-category}
- \wref{2-condensation in a 2-category}

\end{document}
