\documentclass{gwiki}

\Title{Natural Transformation}
\Tags{category-theory, definition}
\Summary{A natural transformation compares two functors by providing componentwise morphisms that commute with every arrow in the source category, so one functor transforms into the other uniformly.}

\begin{document}

\KeySources{\nlab{natural transformation}, \wref{functor}}

\NoteHeader

\begin{idea}[Idea]
Natural transformations are the morphisms between functors: they assign component maps that commute with every arrow in the source category, making the transformation uniform and functorial.
\end{idea}

\prereq{functor, category}

\begin{definition}[Natural Transformation]
Let $F, G : \cC \to \cD$ be \wref{functor}[functors]. A \textbf{natural transformation} $\eta : F \Rightarrow G$ assigns to each object $A \in \cC$ a morphism $\eta_A : F(A) \to G(A)$ such that for every $f : A \to B$ in $\cC$:
\[
\begin{tikzcd}
  F(A) \ar[r, "\eta_A"] \ar[d, "F(f)"'] & G(A) \ar[d, "G(f)"] \\
  F(B) \ar[r, "\eta_B"'] & G(B)
\end{tikzcd}
\]
commutes. The morphisms $\eta_A$ are called the \emph{components} of $\eta$.
\end{definition}

\section{Examples}

\begin{example}[Double dual]
For finite-dimensional vector spaces, there is a natural isomorphism $V \to V^{**}$ given by $v \mapsto (\varphi \mapsto \varphi(v))$.
\end{example}

\begin{example}[Determinant]
$\det : \GL_n \Rightarrow (-)^\times$ is a natural transformation of functors $\Ring \to \Grp$.
\end{example}

\SeeAlso{functor,category}

\References

\Footer

\end{document}
