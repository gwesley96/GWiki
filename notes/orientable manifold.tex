\documentclass{gwiki}
\usepackage{gwiki-links}

\Title{orientable manifold}
\Tags{}

\begin{document}
\NoteHeader

Let $M$ be a topological $n$-\wref[manifold]{topological manifold}. The \textbf{orientation cover} of $M$ is the \wref[double covering]{topology/covering space}
```tikz
\usepackage{amsmath}
\usepackage{amssymb}
\usepackage{tikz-cd}
\begin{document}
\LARGE
\begin{tikzcd}[ampersand replacement=\&,cramped] {\widetilde M} \& {\{(x,\mu_x)\mid x\in M,\mu_x\text{ is a generator of }H_n(M,M\setminus x)\cong\mathbb{Z}\}} \\ M \arrow["{=}"{marking, allow upside down}, draw=none, from=1-1, to=1-2] \arrow["\pi"', two heads, from=1-1, to=2-1] \end{tikzcd}
\end{document}
```
An \textbf{orientation} on $M$ is a \wref{section} $\mu$ of this cover, that is, a map $M\to\widetilde M$ such that $\pi\circ\mu=\mathrm{id}_M$. An \textbf{oriented manifold} is a pair $(M,\mu),$ where $\mu$ is an orientation on the manifold $M.$ A manifold $M$ is called \textbf{orientable} if it admits an orientation $\mu$.

If $M$ is in fact \wref[smooth]{smooth manifold}, then one can show $M$ is \textbf{oriented} if and only if it admits a \wref[smooth atlas]{atlas} $\{(U_\alpha,\phi_\alpha)\}$ for which $\det(\mathrm{d}\phi_\alpha\circ\phi_\beta^{-1})>0$ for all $\alpha,\beta.$ If $M$ is oriented, there is a canonical identification $\mathrm{Dens}^{1}(M)\cong \Lambda^{n}T^{\textit{}M$ of the \wref[density bundle]{density} with $M$; as $M$ is orientable if and only if $\Lambda^{n}T^{}}M$ has a nowhere-vanishing section. (Any \wref{Riemannian metric} yields a (nowhere-vanishing) \wref[positive density]{density}, namely the \wref{Riemannian volume form}.)

\textbf{Remark.} If a smooth manifold $M$ is simply connected, or even if there are no subgroups of index $2$ in $\pi_1(M),$ then $M$ is orientable. Thus, for instance, simply connected smooth manifolds are orientable.




\textbf{Example.} The boundary of an oriented \wref[smooth manifold]{smooth manifold} is an oriented smooth manifold. Indeed,
$$\frac{\mathrm{d}}{\mathrm{d}x}(\phi_{\alpha}\circ\phi_{\beta})= \left( \begin{array}{c|ccc} * & 0 & \cdots & 0 \\  \hline 0 & & & \\ \vdots& & \frac{\mathrm{d}}{dx}(\phi_{\alpha}^{\partial} \circ (\phi_{\beta}^{\partial})^{-1}) & \\ 0 & & & \end{array} \right).$$
By hypothesis $\det(\delta(\phi_\alpha^\partial \circ\phi_\beta^{-1}))>0$, so since the determinant of a block matrix is multiplicative, it suffices to show $*>0,$ i.e. that $\frac{\partial(\phi_\alpha\circ\phi_\beta^{-1})}{\partial x_1}>0.$ (Why? Because $\mathbb{R}_+^n$ is invariant.) $\square$ 

See also:
- [Here](https://math.ucr.edu/~res/m205C/orientation001.pdf)

\end{document}
