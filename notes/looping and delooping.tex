\documentclass{gwiki}

\Title{looping and delooping}
\Tags{}

\begin{document}

\NoteHeader

#### In $(\infty,1)$-categories
Let $C$ be an \wref{(∞,1)-category} with a \wref[terminal object]{terminal object in an (∞,1)-category} $*$.

For any \wref{pointed object} $X$ in $C$, its **loop space object** (AKA **looping**) of $X$, denoted $\Omega X$, is defined as the \wref[(∞,1)-pullback]{(∞,1)-pullback in an (∞,1)-category} of the point $x_{0}$ along itself in $C$:
```tikz
\usepackage{amsmath}
\usepackage{amssymb}
\usepackage{tikz-cd}
\begin{document}
\huge
\begin{tikzcd}
\Omega X \arrow[r] \arrow[d] & * \arrow[d, "x_0"] \\
* \arrow[r, "x_0"'] & X
\end{tikzcd}
\end{document}
```
A **delooping** of an object $A\in C$ is a pointed object $\mathrm{B}A$ such that there is an \wref[equivalence]{equivalence in an (∞,n)-category} $A \simeq \Omega(\mathrm{B}A)$ in $C$.

#### In $(\infty,n)$-categories

See \wref[Remark 4.4.6]{LurieTQFTClassification2009.pdf#page=160&selection=133,0,133,12&color=yellow}.

\References

\Footer

\end{document}
