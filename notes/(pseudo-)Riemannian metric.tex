% ---
% title: (pseudo-)Riemannian metric
% tags:
%   - differential-geometry
%   - riemannian-geometry
%   - metric-geometry
% aliases:
%   - pseudo-inner product space
%   - pseudo-Euclidean vector space
%   - Euclidean metric
%   - Poincaré metric
%   - Poincaré disk model
%   - hyperbolic space
%   - pseudo-Riemannian manifold
%   - Riemannian manifold
% date created: August 9, 2025 at 9:12 pm ET
% date modified:
%   - November 11, 2025 at 10:19 pm ET
%   - August 16, 2025 at 5:28 pm ET
% ---
\documentclass{gwiki}
\usepackage{tz}

\Title{(Pseudo-)Riemannian Metric}
\Tags{differential-geometry}{riemannian-geometry}{metric-geometry}
\Aliases{pseudo-inner product space,pseudo-Euclidean vector space,Euclidean metric,Poincaré metric,Poincaré disk model,hyperbolic space,pseudo-Riemannian manifold,Riemannian manifold}
\Summary{Positive-semidefinite and positive-definite quadratic data on vector spaces and manifolds, including canonical models.}
\begin{document}

\NoteHeader

\begin{idea}[Idea]
Metrics capture the quadratic data needed to measure lengths and angles; pseudo-Riemannian metrics relax positivity to accommodate signatures such as Lorentzian geometries while retaining smooth variation over the manifold.
\end{idea}

\begin{definition}[Pseudo-inner product]
A \textbf{pseudo-inner product} (resp. \textbf{inner product}) on a vector space $V$ is a \wref[positive-semidefinite form]{Positive Definite Form} (resp. \wref[positive-definite form]{Positive Definite Form}) symmetric bilinear form $\bkt{-}{-}\colon V\times V\to\bbR$.
\end{definition}

\begin{definition}[(Pseudo-)Euclidean vector spaces]
A \textbf{pseudo-Euclidean vector space} (resp. \textbf{Euclidean vector space}) is a finite-dimensional $\bbR$-vector space $V$ equipped with a positive-semidefinite (resp. positive-definite) \wref[quadratic form]{Quadratic Form} $Q\colon V\to\bbR$.
\end{definition}

\begin{remark}[(Pseudo-)norms from quadratic forms]
A (pseudo-)Euclidean vector space $(V,Q)$ has \textbf{(pseudo-)norm} $\|-\|\coloneqq \sqrt{Q(-)}$.
\end{remark}

\section{(Pseudo-)Riemannian metrics}

\begin{definition}
A \textbf{pseudo-Riemannian metric} on a smooth manifold $M$ is a $(0,2)$-\wref[tensor field]{Tensor Field} $\mathrm{g}$ such that for each $p\in M$, the bilinear form $\mathrm{g}_p\coloneqq \bkt{-}{-}_p\colon T_pM\times T_pM\to \bbR$ is symmetric and nondegenerate. If $\mathrm{g}$ is positive-definite, it is a \textbf{Riemannian metric}.
\end{definition}

\begin{definition}[(Pseudo-)Riemannian manifolds]
A \textbf{(pseudo-)Riemannian manifold} is a \wref[smooth manifold]{Smooth Manifold} $M$ equipped with a (pseudo-)Riemannian metric $\mathrm{g} \coloneqq \bkt{-}{-}$.
\end{definition}

\begin{example}[Euclidean metric]
For $M = \bbR^n$ with global coordinates $x^i$, the \textbf{Euclidean metric} $\overline{\mathrm{g}} \coloneqq (\mathrm{d} x^1)^2 + \cdots + (\mathrm{d} x^n)^2$ makes $\bbR^n$ into a Riemannian manifold, the standard Euclidean vector space of dimension $n$.
\end{example}

\begin{example}[Poincaré disk model]
For $M = \overset\circ D{}^n \subset \bbR^n$, the \textbf{Poincaré metric} $\mathrm{g}_{\small\mathbb{H}}\coloneqq \dfrac{4\overline{\mathrm{g}}}{1-\overline{\mathrm{g}}}$ yields the Poincaré disk model of $n$-dimensional \textbf{hyperbolic space} $\mathbb{H}^n$.
\end{example}

\begin{framedproposition}[{(Lee18, Prop. 2.4)}]
Every smooth manifold admits a Riemannian metric; one proof uses a \wref{Partition of Unity}.
\end{framedproposition}

\SeeAlso{Smooth Manifold,Quadratic Form,Tensor Field}
\IncomingLinks{DG Algebra}

Further reading: \nlab{pseudo-Riemannian manifold}.
\References

\Footer

\end{document}
