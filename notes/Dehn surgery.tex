\documentclass{gwiki}
\usepackage{gwiki-links}

\Title{Dehn surgery}
\Tags{}

\begin{document}
\NoteHeader

\textbf{Idea.} Dehn surgery is a method for constructing one \wref[manifold]{manifolds as ringed spaces} from another by a certain kind of \textit{cut-and-paste} procedure, hence the name. 

Particularly useful is surgery on \wref[3-manifolds]{3-manifold},  pair $(L,p/q)$ where $L$ is a \wref{framed link} and $p/q\in\mathbb{Q}$. By a Theorem of Lickorish–Wallace, every \wref[connected]{connected topological space} \wref[oriented]{orientable manifold} \wref{3-manifold} $X$ arises in this way, and in fact we can choose such a presentation for $X$ where $p/q\in\mathbb{Z}$. For a great introductory video of this, see the YouTube video \href{https://www.youtube.com/watch?v=govsh086h1A}{here}.

\textbf{Construction (Dehn surgery along a framed link $L$ in $S^{3}$).} See \href{https://ncatlab.org/nlab/show/Dehn\%2Bsurgery?utm_source=chatgpt.com#method}{here}.




See also:
- \wref{Kirby calculus}
- \wref{Dehn twist}

\end{document}
