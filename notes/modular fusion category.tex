\documentclass{gwiki}
\usepackage{gwiki-links}

\Title{modular fusion category}
\Tags{}

\begin{document}
\NoteHeader

\textbf{Idea.} \wref[Although it is not precise, it is still helpful to think of a modular category as a categorification of a Frobenius algebra.]{BartlettTQFTs.pdf#page=84&selection=98,60,99,44}

A \textbf{modular fusion category} (MFC) (AKA \textbf{modular tensor category} (MTC)) is a \wref{ribbon fusion category} satisfying any (hence all) of the following \wref[equivalent conditions]{EGNO15.pdf#page=260&selection=132,0,134,7}.
- $\mathcal{C}$ is \textbf{nondegenerate}, i.e., the \wref{S-matrix} of $\mathcal{C}$ is invertible.
- The \wref{symmetric center} $\mathcal{C}'$ of $\mathcal{C}$ is trivial ($\mathcal{C}'\cong\mathsf{Vec}$).
- $\mathcal{C}$ is \textbf{factorizable}, i.e., the \wref{braided monoidal functor} $\mathcal{C}\boxtimes \mathcal{C}^{\mathrm{rev}}\to\mathcal{Z}(\mathcal{C})$ given by the Deligne product of the functors $c\mapsto (c,\beta_{c,-})$ and $c\mapsto (c,\beta_{-,c})$ is an equivalence (where $\mathrm{rev}$ denotes $\mathcal{C}$ with braiding $\beta_{x,y}^{\mathrm{rev}}\coloneqq \beta_{y,x}^{^{-1}}$).
\textbf{\wref[Proposition 8.20.12]{EGNO15.pdf#page=260&selection=132,0,134,7}.}

\textbf{\wref[Theorem]{ColDelThesisQuantumSymmetries.pdf#page=18&selection=51,0,51,7} ([here](https://arxiv.org/abs/1509.06811)).} A modular tensor category is equivalent to a 3-2-1 \wref[TQFT]{TQFT}.
!\wref{ColDelThesisQuantumSymmetries.pdf#page=19&rect=91,431,536,683}

\textbf{\wref[Important Remark]{ColDelThesisQuantumSymmetries.pdf#page=31}.} A UMTC is fundamentally a category whose objects can be interpreted as collections of anyons and morphisms as anyon processes, together with three coherent structures with precisely the right properties to enforce the \wref{axioms of a quantum system}:
1) the monoidal structure to define quantum states with properties that allow these states to be superposed and entangled, the pivotal structure to define measurement; 
2) the structure of a braiding to describe evolution of states; and 
3) unitarity to ensure unitary time evolution.

\subsection*{Examples}
See \wref[here]{ColDelThesisQuantumSymmetries.pdf#page=59&selection=6,0,6,8} for several examples and their data.

See also: 
- \wref{anyon}
- \wref[Fibonacci topological order]{Fibonacci category}
- \wref{Bulk–Boundary Correspondence}

\end{document}
