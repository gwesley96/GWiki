\documentclass{gwiki}

\Title{Gelfand transform}
\Tags{}

\begin{document}

\NoteHeader

For a commutative \wref{banach-algebra} $A$, the \textbf{Gelfand transform}, given by
\[
\begin{aligned}
A&\to C(\widehat{A}),\\
a&\mapsto[\operatorname{ev}_a\colon \varphi\mapsto\varphi(a)],
\end{aligned}
\]
is a norm-decreasing unital algebra homomorphism, where $\widehat{A}$ is the set of \textbf{characters} (AKA \textbf{multiplicative linear functionals}) on $A$, that is, of nonzero algebra homomorphisms $A\to\mathbb{C}$.

The image of the Gelfand transform is a subalgebra of $C(\widehat{A})$ that separates points of $\widehat{A}$.

If $A$ is a commutative unital Banach algebra and $a\in A$, then for all $\varphi\in\widehat{A}$, $\varphi(a)\in\operatorname{sp}(a)$.

See also:
\begin{itemize}
\item \wref{spectrum-(banach-algebra)}
\end{itemize}

\Footer

\end{document}

