Outline (to be filled in later): • A (disk-like) 2-category with only one object is what’s known as a spherical tensor category. (Need to give details of equivalence and full definition of spherical category.) • Given a spherical category C, with objects C0 and morphisms C1, we can define fields on surfaces consisting of embedded graphs with oriented edges labeled by C0 and vertices (“coupons” in R-T terminology) labeled by C1. (Graphs are allowed to have circular edges without vertices.) Frobenius reciprocity means that we don’t need to distinguish range and domain at the coupons. Fields on a 1-manifold are oriented points (ends of oriented arcs), labeled by C0. There is a unique (empty) field on a 0-manifold.

\begin{example}[{\wref[Spherical categories]{WalkerTQFTnotes06.pdf\#page=74\&annotation=1667R}}] The local relations are generated by
\begin{lst} 
\item isotopy, 
\item reversing the orientation of an \emph{arc} and changing the label to its dual, 
\item replacing “identity” coupons with parallel \emph{arcs} (note that this includes cups and caps because of Frobenius reciprocity), 
\item erasing \emph{arcs} labeled by the trivial object (and adjusting labels of coupons if necessary), 
\item combining two adjacent coupons into one (using composition of morphisms in $\cC$), and  
\item replacing a diagram containing a coupon labeled by a linear combination of morphisms with the corresponding linear combination of fields.
\end{lst}
\end{example}

For this theory, A(pt) is essentially C (thought of as a 2-category, of course). (Need to give more details on “essentially”.) • A(S1) is the annularization of C. • Z(S1) is Drinfeld center of the category of representations of C. (See (5.2.30).) • Note that any labeled graph in S2 is equivalent, via the above local relations, to some multiple of the empty graph. We call this multiple the “standard evaluation” of a graph in S2. • A(Y 2) can be described in terms of labelings of 0- and 1-skeleton of a fixed8.2.1 cell decomposition of the surface Y . An explicit list of relations corresponding to 2-cells of the cell decomposition can be given. (Give details.) Note that we already know that the vector space is independent of the cell decomposition, so we don’t need a separate proof of independence. • If C is semisimple, then we can restrict all coupons to be trivalent. We can also restrict all edge labels to be simple objects (or minimal idempotents or irreps; need to comment on equivalence between minimal idems, irreps and simple objects). We get the familiar description in terms of labeled, oriented trivalent graphs. (Plain “trivalent graphs” for short.) • (Need to give details on “F” moves (a.k.a. recoupling), etc.) • (Give refinement of description of A(Y ; c), including case when Y is a disk.8.2.2 Observe that in this case we have an orthogonal basis.) Assume now that C is s semisimple spherical category with finitely many irreps. It follows from (8.2.1) that A(Y ; c) is finite dimensional for all 2-manifolds Y and c ∈ C(∂Y ). Note that Z(S2) is 1-dimensional, and the standard evaluation of trivalent graphs (which evaluates to 1 ∈ C on the empty graph) is a basis. It follows G 9/16/2025, 8:54:06 PM 8.2. SPHERICAL CATEGORIES IN GENERAL 71 from (8.2.2) that any non-zero element z ∈ Z(S2) determines a nondegenerate inner product on A(D2; c) for all c. Assume that these inner products are positive definite. [comment on this assumption.] We can now apply (6.3.1) to construct a path integral for 3-manifolds. Following the proof of (6.3.1) we will construct a state sum description of the path integral in terms of labelings of the cells of a cell decomposition of a 3-manifold. We will see that this state model turns out to be the Turaev-Viro state model [TV92, Tur94].

\SeeAlso{ribbon category,local relations of a system of topological fields,system of topological fields}