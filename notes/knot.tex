\documentclass{gwiki}
\usepackage{gwiki-links}

\Title{knot}
\Tags{}

\begin{document}
\NoteHeader

A \textbf{knot} in $\mathbb{R}^{3}$ (or equivalently $S^{3}$) is a \wref[smooth embedding]{embedding} $S^{1}\hookrightarrow \mathbb{R}^{3}$. More generally, a \textbf{link} is a smooth embedding $\coprod_{1}^nS^{1}\hookrightarrow \mathbb{R}^{3}$ for some $n\in\mathbb{Z}_{\geq 0}$. 

Thus a link is a disjoint union of knots that can be "knotted" among themselves.!\wref{UQSLPartIChapter4.pdf#page=31&rect=68,472,543,611}
We typically represent a knot in $\mathbb{R}^{3}$ by a \textbf{knot/link projection} to $\mathbb{R}^{2}$, that is, by its image in $\mathbb{R}^{2}$ under a \textbf{generic regular projection}, that is, a projection $\pi\colon \mathbb{R}^{3}\to\mathbb{R}^{2}$ that avoids various "bad" behaviors, such as triple intersections and kinks. Knot projections of knots in $\mathbb{R}^{3}$ are classified by \wref{Reidemeister's Theorem}.

By \wref{Markov's Theorem}, every link is the closure of some \wref{braid}.

See also:
- \wref{braid}

\end{document}
