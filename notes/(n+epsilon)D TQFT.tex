\documentclass{gwiki}

\Title{$(n+\varepsilon)$D TQFT}
\Tags{TQFT, skein-modules, higher-categories}
\Topics{Definition of $(n+\varepsilon)$-dimensional TQFTs via systems of topological fields and local relations, and the construction of skein modules and skein categories.}

\begin{document}

\NoteHeader

\begin{definition}[$(n+\varepsilon)$-dimensional TQFT]
An \textbf{$(n+\varepsilon)$-dimensional TQFT} $(\mathcal{F},\mathcal{U})$ is just an $n$\wref{system of topological fields}[-dimensional system of topological fields] and \wref{extended isotopy}[local relations].
\end{definition}

The reason for this terminology is that although an $(n+\varepsilon)$-dimensional TQFT does not have the data of the path integral/partition function of a typical $(n+1)$-dimensional manifold, it does have this data for \wref{mapping class group}[mapping cylinders] between $n$-manifolds.

\section{Skein Modules and Categories}

\begin{construction}[Skein module from an $(n+\varepsilon)$-dimensional TQFT]\label{const:skein-module}
Let $(\mathcal{F},\mathcal{U})$ be an $(n+\varepsilon)$D TQFT and consider an $n$-manifold $X$ with boundary condition $c\in\mathcal{F}(\partial X)$. Let $R$ be any unital \wref[ring]{ring (nPOV)}. We define the \textbf{skein module} associated to $(X,c)$ by the free $R$-module generated by fields on $X$ (in this context, often called base fields) modulo local relations on $(X,c)$, i.e.,
\[
  \mathrm{Sk}_{\mathcal{F},\mathcal{U}}(X,c;R)\coloneqq R[\mathcal{F}(X,c)]/\mathcal{U}(X,c).
\]
By a slight abuse of terminology, we will often refer to elements of $\mathrm{Sk}_{\mathcal{F},\mathcal{U}}(X,c;R)$ again as fields.
\end{construction}

More generally, for any $1 \leq k \leq n$ we can consider $(W,c)$ for any $(n - k)$-manifold $W$ and any boundary condition $d\in\mathcal{F}(\partial Y)$. We define the \textbf{skein $k$-category} $\mathrm{Sk}_{\mathcal{F},\mathcal{U}}(W;d;R)$ to be the following (traditional, algebraic, linear) $R$-linear $k$-category whose morphisms we define recursively as follows.
\begin{itemize}[nosep]
  \item \textbf{0-morphisms.} The objects of $\mathrm{Sk}_{\mathcal{F},\mathcal{U}}(W,d;R)$ are elements of the subset $\mathcal{F}(W;d)\coloneqq \partial^{-1}(d)$ of $\mathcal{F}(W)$.
  \item \textbf{$j$-morphisms for $1\leq j\leq n-1$.} For any two compatible $(j-1)$-morphisms $a,b$ in $\mathrm{Sk}_{\mathcal{F},\mathcal{U}}(W,d;R)$, we set $\mathrm{Sk}_{\mathcal{F},\mathcal{U}}(W,d;R)(a\to b)\coloneqq \mathcal{F}(W\times D^j,(\widehat a, b);R)$.
  \item \textbf{$n$-morphisms.} For any two compatible $(n-1)$-morphisms $a,b$ in $\mathrm{Sk}_{\mathcal{F},\mathcal{U}}(W,d;R)$, we set $\mathrm{Sk}_{\mathcal{F},\mathcal{U}}(W,d;R)(a\to b)\coloneqq \mathrm{Sk}_{\mathcal{F},\mathcal{U}}(W\times D^j,(\widehat a, b);R)$.
\end{itemize}
As usual, in this construction, unless $\partial W=\varnothing$, every product $W\times D^j$ has a ``pinched'' boundary so that $\partial (W\times D^{j})\cong \Delta W\coloneqq \overline{W}\cup_{\partial W} W$, the \wref{double of a manifold}[double] of $Y$.

(Note that we have not showed how $j$-composition works in general, but composition is given by gluing.)

We will be more interested in the dual space (representations) of $\mathrm{Sk}_{\mathcal{F},\mathcal{U}}(X,c;R)$, which we denote by $Z_{\mathcal{F},\mathcal{U}}(X,c;R)$. This is because $Z_{\mathcal{F},\mathcal{U}}(X,c;R)$ turn out to have nicer algebraic structure than the $\mathrm{Sk}_{\mathcal{F},\mathcal{U}}(X,c;R)$-invariants; in particular, at least for $n=1,2$, the former are always \wref[Cauchy complete]{Cauchy complete n-category} even when the latter are not.

\textbf{Notation.} Often the system of fields and local relations $(\mathcal{F},\mathcal{U})$ that define $\mathrm{Sk}_{\mathcal{F},\mathcal{U}}$ and $Z_{\mathcal{F},\mathcal{U}}$ are should be clear from context, and we simply write $\mathrm{Sk}$ and $Z$ respectively.

\section{Examples}

Our main example of an $n$-dimensional system of fields will be $\mathcal{C}$-labeled string diagrams for some ``traditional $n$-category with strong duality'' $\mathcal{C}$.

Recall that a $\mathcal{C}$-\textbf{labeled string diagram} on a $k$-manifold $Y$ is an embedded cell complex of $Y$ whose codimension-$j$ cells are labeled by $j$-morphisms of $\mathcal{C}$. (We will give the precise definition of a $\mathcal{C}$-labeled string diagram after providing the full definition of a system of fields.) It will be helpful to think of string diagrams as the ``fields'' in the following definition.

\section{(Unitary) $n$-Categories with Strong Duality}

% TODO: expand this section

\SeeAlso{(n+1)D TQFT,system of topological fields,local relations for a system of topological fields,extended isotopy}

\References

\Footer

\end{document}
