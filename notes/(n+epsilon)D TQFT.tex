\documentclass{gwiki}

\Title{$(n+\varepsilon)$D TQFT}
\Tags{disk-like framework, TQFT, skein-modules, higher-categories}
\Topics{Definition of $(n+\varepsilon)$-dimensional TQFTs via systems of topological fields and local relations, and the construction of skein modules and skein categories.}

\newcommand{\A}{\mathrm{Sk}}
\newcommand{\Z}{Z}
\newcommand{\M}{\dM}
\newcommand{\F}{\eF}
\newcommand{\orev}{\overline}
\newcommand{\fcj}{\widehat}
\renewcommand{\U}{\eU}
\newcommand{\undC}{\smash[b]{\text{$\underset{\smash{\raisebox{.275em}{$\scriptstyle\kern.1ex\rightarrow$}}}{\smash{\cC}}$}}}

\begin{document}

\NoteHeader

\begin{definition}[$(n+\varepsilon)$-dimensional TQFT]
An \textbf{$(n+\varepsilon)$-dimensional TQFT} $(\F,\U)$ is just an $n$\wref{system of topological fields}[-dimensional system of topological fields] and \wref{extended isotopy}[local relations].
\end{definition}

The reason for this terminology is that although an $(n+\varepsilon)$-dimensional TQFT does not have the data of the path integral/partition function of a typical $(n+1)$-dimensional manifold, it does have this data for \wref{mapping class group}[mapping cylinders] between $n$-manifolds.

\section{Skein modules and skein $k$-categories}

\begin{construction}[Skein module from an $(n+\varepsilon)$-dimensional TQFT]\label{const:skein-module}
Let $(\F,\U)$ be an $(n+\varepsilon)$D TQFT and consider an $n$-manifold $X$ with boundary condition $c\in\F(\partial X)$. Let $R$ be any unital \wref[ring]{ring (nPOV)}. We define the \textbf{skein module} associated to $(X,c)$ by the free $R$-module generated by fields on $X$ (in this context, often called base fields) modulo local relations on $(X,c)$, i.e.,
\[
  \A_{\F,\U}(X,c;R)\coloneqq R[\F(X,c)]/\U(X,c).
\]
By a slight abuse of terminology, we will often refer to elements of $\A_{\F,\U}(X,c;R)$ again as fields.
\end{construction}

\begin{construction}[Skein $k$-category from an $(n+\e)$D TQFT]
More generally, for any $1 \leq k \leq n$ we can consider $(W,c)$ for any $(n - k)$-manifold $W$ and any boundary condition $d\in\F(\partial Y)$. We define the \textbf{skein $k$-category} $\A_{\F,\U}(W;d;R)$ to be the following (traditional, algebraic, linear) $R$-linear $k$-category whose morphisms we define recursively as follows.
\begin{itemize}
  \item \textbf{0-morphisms.} The objects of $\A_{\F,\U}(W,d;R)$ are elements of the subset $\F(W;d)\coloneqq \partial^{-1}(d)$ of $\F(W)$.
  \item \textbf{$j$-morphisms for $1\leq j\leq n-1$.} For any two compatible $(j-1)$-morphisms $a,b$ in $\A_{\F,\U}(W,d;R)$, we set $\A_{\F,\U}(W,d;R)(a\to b)\coloneqq \F(W\times D^j,(\fcj a, b);R)$.
  \item \textbf{$n$-morphisms.} For any two compatible $(n-1)$-morphisms $a,b$ in $\A_{\F,\U}(W,d;R)$, we set $\A_{\F,\U}(W,d;R)(a\to b)\coloneqq \A_{\F,\U}(W\times D^j,(\fcj a, b);R)$.
\end{itemize}
As usual, in this construction, unless $\partial W=\varnothing$, every product $W\times D^j$ has a ``pinched'' boundary so that $\partial (W\times D^{j})\cong \Delta W\coloneqq \overline{W}\cup_{\partial W} W$, the \wref{double of a manifold}[double] of $Y$. (Note that we have not showed how $j$-composition works in general, but composition is given by gluing.)
\end{construction}

For the rest of this article we take $R$ to be the complex numbers $\mathbb{C}$, so we will henceforth drop $R$ from the notation.

We will be more interested in the dual space (representations) of $\A_{\F,\U}(X,c)$, which we denote by $Z_{\F,\U}(X,c)$. This is because $Z_{\F,\U}(X,c)$ turn out to have nicer algebraic structure than the $\A_{\F,\U}(X,c)$-invariants; in particular, at least for $n=1,2$, the former are always \wref[Cauchy complete]{Cauchy complete n-category} even when the latter are not.

\begin{notation} 
Often the system of fields and local relations $(\F,\U)$ that define $\A_{\F,\U}$ and $Z_{\F,\U}$ are should be clear from context, and we simply write $\A$ and $Z$ respectively.
\end{notation}

\section{Examples}
Our main example of an $n$-dimensional system of fields will be $\cC$-labeled string diagrams for some ``traditional $n$-category with strong duality'' $\cC$.

% Recall that a $\cC$-\textbf{labeled string diagram} on a $k$-manifold $Y$ is an embedded cell complex of $Y$ whose codimension-$j$ cells are labeled by $j$-morphisms of $\cC$. (We will give the precise definition of a $\cC$-labeled string diagram after providing the full definition of a system of fields.) It will be helpful to think of string diagrams as the ``fields'' in the following definition.

\SeeAlso{(n+1)D TQFT,unitary (n+1)D TQFT,system of topological fields,local relations for a system of topological fields,extended isotopy,disk-like n-category}

\References

\Footer

\end{document}
