\documentclass{gwiki}
\usepackage{gwiki-links}

\Title{Riemann (0,4)-curvature tensor}
\Tags{}

\begin{document}
\NoteHeader

By raising/lowering indices, we can convert the \wref{Riemann (1,3)-curvature tensor} $R$ to a (0,4)-tensor field $\mathrm{Rm}$, called the \textbf{Riemann curvature} (AKA \textbf{Riemann (0,4)-curvature tensor}), which is thus given by$$\mathrm{Rm}(u,v,w,q)\coloneqq \braket{{R(u,v)w}|{q}}\qquad\forall u,v,w,q\in\mathfrak{X}(M).$$
See also: 
- \wref{Algebraic Bianchi Identity}
- \wref{Riemann (0,4)-curvature tensor}
- \wref{Ricci tensor}
- \wref{Ricci curvature}
- \wref{scalar curvature}

\end{document}
