\documentclass{gwiki}
\usepackage{gwiki-links}

\Title{quadratic form}
\Tags{}

\begin{document}
\NoteHeader

For a $\mathbb{R}$-vector space $V$, a \textbf{quadratic form} on $V$ is a is a linear map $Q\colon V\to \mathbb{R}$ such that $Q(\lambda v)=\lambda^2v$ for all scalars $\lambda$ and $(v,w)\mapsto Q(v+w)-Q(v)-Q(w)$ is bilinear. Equivalently, a quadratic form on $V$ is a homogeneous polynomial of degree $2$ in $n\coloneqq \dim V$ variables, i.e., $Q(x)=\sum_{i,j=1}^na_{ij}x_ix_j$ for some $a_{ij}\in K$.

Let $Q\colon V\to \mathbb{R}$ be a quadratic form. We call $Q$ \textbf{positive-semidefinite} (resp. \textbf{positive-definite}) if $Q(v)\geq 0$ (resp. if $Q(v)\geq 0$ with equality if and only if $v=0$). The conditions to call $Q$ \textbf{negative-semidefinite} and \textbf{negative-definite} should now be obvious. We call $Q$ \textbf{indefinite} if there are $v,v'\in V$ such that $Q(v)>0$ and $Q(v')<0$.

\textbf{TODO}: signature of a quadratic form

See also:
\begin{itemize}
\item \wref{symmetric bilinear form}
\end{itemize}

\end{document}
