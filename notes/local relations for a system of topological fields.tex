\documentclass{gwiki}

\Title{local relations for a system of topological fields}
\Tags{}
\newcommand{\gwikicreateddate}{October 28, 2025 at 9:47 am ET}

\begin{document}

\NoteHeader
\NoteNavigation

A \textbf{system of local relations} for an $n$\wref[-dimensional system of topological fields]{system of topological fields}$\mathcal{F}$is a collection of subspaces$\mathcal{U}(D,c) \subset \mathscr{F}(D,c)$, indexed over$n$\wref[-balls]{n-ball}$D$and boundary conditions$c \in \mathscr{F}(\partial D)$that satisfies the following properties.
\begin{lst}
  \item (\textbf{U1}) (\textit{Functoriality})$f_*\mathcal{U}(D,c)=\mathcal{U}(D',f_*c)$for all homeomorphisms$f\colon D \to D'$.
  \item (\textbf{U2}) (\textit{Extended-isotopy invariance}) If$x, y \in \mathscr{F}(D, c)$and$x$is extended-isotopic to$y$, then$x-y\in \mathcal{U}(D,c)$.
  \item (\textbf{U3}) (\textit{Ideal under gluing}) If$D'$is an$n$-ball embedded in$D$(possibly intersecting$\partial D$), then for any$x \in \mathcal{U}(D')$and$r \in \mathscr{F}(D\setminus D')$, we have$x \bullet r \in \mathcal{U}(D)$.
\end{lst}
Now let$X$be any$n$-manifold. For$n$-balls$D$embedded in$X$(possibly intersecting$\partial X$and compatible string diagrams$c \in \mathcal{F}(\partial M)$,$d \in \mathcal{F}(\partial X)$, and$e \in \mathcal{F}(\partial(M \setminus D))$, there is a gluing map$$R[\mathcal{F}(X \setminus D; e)] \otimes_{R} U(D; d) \to R[\mathcal{F}(X; c)].$$Define$U(X; c)$to be the span of the images of the above gluing maps, for all such$B$,$d$, and$e$.

\subsection{Examples}
\wref[Example (Spherical categories).]{WalkerTQFTnotes06.pdf\#page=74\&annotation=1667R} The local relations are generated by 1. isotopy, 2. reversing the orientation of an arc and changing the label to its dual, 3. replacing “identity” coupons with parallel arcs (note that this includes cups and caps because of Frobenius reciprocity), 4. erasing arcs labeled by the trivial object (and adjusting labels of coupons if necessary), 5. combining two adjacent coupons into one (using composition of mor- phisms in  C), and 6. replacing a diagram containing a coupon labeled by a linear combination of morphisms with the corresponding linear combination of fields.

\wref[Example (Ribbon categories).]{WalkerTQFTnotes06.pdf\#page=79\&annotation=1679R} The local relations are generated by 1. isotopy, 2. reversing the orientation of a ribbon and changing the label to its dual, 3. replacing “identity” coupons with parallel ribbons (note that this includes cups and caps because of Frobenius reciprocity), 4. erasing ribbons labeled by the trivial object (and adjusting adjacent labels of coupons as necessary), 5. combining two adjacent coupons into one (using composition of mor- phisms in  C), and 6. replacing a diagram containing a coupon labeled by a linear combination of morphisms with the corresponding linear combination of fields.

See also:
\begin{lst}
  \item \wref{(n+epsilon)D TQFT}
\end{lst}

\References

\Footer

\end{document}
