
Outline (to be filled in later): • Note that this section is very similar to Section 8.2, so the reader might want to compare the two, or treat Section 8.2 as a warm-up for this section. • A disklike 3-category with only one object (0-morphism) and only one 1morphism is a ribbon category. [need to give details] Thus we expect that a from a ribbon category we can construct a 3+1-dimensional TQFT. We will see in the next section that (with some additional assumptions on the ribbon category) this 3+1-dimensional TQFT can be decategorified to yield the Witten-Reshtikhin-Turaev type 2+1-dimensional TQFT that can be constructed directly (but less cleanly) from the same ribbon category. • Given a ribbon category C, with objects C0 and morphisms C1, we can define fields on 3-manifolds consisting of embedded ribbon graphs with oriented edges labeled by C0 and vertices labeled by C1. Following Reshetikhin and Turaev, we will call a labeled vertex a coupon. [need to comment on pinning or cyclic ordering or ciliation at vertices] (Graphs are allowed to have circular edges without vertices.) The dualities of the ribbon category mean that we don’t need to distinguish range and domain at the coupons. Fields on a 2-manifold are oriented framed points (ends of oriented ribbons), labeled by C0. There is a unique (empty) field on a 1-manifold. There is a unique (empty) field on a 0-manifold.


\begin{example}[{\wref[Ribbon categories]{WalkerTQFTnotes06.pdf\#page=79\&annotation=1679R}}]
The local relations are generated by 
\begin{lst} 
\item isotopy, 
\item reversing the orientation of a \emph{ribbon} and changing the label to its dual,
\item replacing “identity” coupons with parallel \emph{ribbons} (note that this includes cups and caps because of Frobenius reciprocity), 
\item erasing \emph{ribbons} labeled by the trivial object (and adjusting adjacent labels of coupons as necessary), 
\item combining two adjacent coupons into one (using composition of morphisms in $\cC$), and
\item replacing a diagram containing a coupon labeled by a linear combination of morphisms with the corresponding linear combination of fields.
\end{lst}
\end{example}


• For this theory, A(pt) is essentially C (thought of as a 3-category, of course). [Need to give more details on “essentially”.] • Note that any labeled ribbon graph in (B3, ∅) is equivalent, via the above local relations, to some multiple of the empty graph. We call this multiple the “standard evaluation” of a graph in (B3, ∅). • A(M 3; c) can be described in terms of labelings of 0- and 1-skeleton of a9.1.1 fixed cell decomposition of the 3-manifold M . An explicit list of relations corresponding to 2-cells of the cell decomposition can be given. [Give details.] • [Need to remark that A(M 3; c) is the familiar (relative) skein module of M based on C.] • If C is semisimple, then we can restrict all coupons to be trivalent. We can also restrict all edge labels to be simple objects [or minimal idempotents or irreps; need to comment on equivalence between minimal idems, irreps and simple objects]. We get the familiar description in terms of labeled, oriented trivalent graphs. (Plain “trivalent graphs” for short.) • (Need to give details on “F” moves (a.k.a. recoupling), etc.) • Let L be a complete set of irreps of C. • (Give refinement of description of A(M ; c), including case when M is a disk.9.1.2 Observe that in this case we have an orthogonal basis.) Assume now that C is s semisimple ribbon category with finitely many irreps. It follows from (9.1.1) that A(M ; c) is finite dimensional for all 3-manifolds M and c ∈ C(∂M ). Note that Z(S3) is 1-dimensional, and the standard evaluation of trivalent ribbon graphs (which evaluates to 1 ∈ C on the empty graph) is a basis. It follows from (9.1.2) that any non-zero element z ∈ Z(S3) determines a nondegenerate inner product on A(B3; c) for all c. Assume that these inner products are positive definite. [comment on this assumption.] We can now apply (6.3.1) to construct a path integral for 4-manifolds.

\SeeAlso{spherical category,local relations of a system of topological fields,system of topological fields}
