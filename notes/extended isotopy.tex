\documentclass{gwiki}
\usepackage{gwiki-links}

\Title{extended isotopy}
\Tags{}

\begin{document}
\NoteHeader

\textbf{Definition (Extended isotopy.} Let $X$ be an $n$-manifold and $Y \subset \partial X$ be a codimension-0 submanifold of $\partial X$. Let $X \cup_Y (Y \times I)$ denote $X$ glued along $Y$ to the pinched product $Y \times I$. Extend the product structure on $Y \times I$ to a bicollar neighborhood of $Y$ inside $X \cup_Y (Y \times I)$. We call a homeomorphism$$f\colon  X \cup_Y (Y \times I) \overset\sim\to X$$a \textbf{collaring homeomorphism} if $f$ is the identity outside of the bicollar and $f$ preserves the fibers of the bicollar. The maps $\mathscr{F}(X) \to \mathscr{F}(X)$ that collar homeomorphisms induce via the functoriality and product field properties above are called \textbf{collar maps}. Let $X$ and $Y \subset \partial X$ be as above. Let $x \in \mathscr{F}(X)$ be a field on $X$ and such that $\partial x$ is splittable along $\partial Y$. Let $c$ be $x$ restricted to $Y$. Then we have the glued field $x \bullet (c \times I)$ on $X \cup (Y \times I)$. Let $f\colon  X \cup (Y \times I) \to X$ be a collaring homeomorphism. Then we call the map $x \mapsto f(x \bullet (c \times I))$ a \textbf{collar map}. 

\textbf{Definition.} We call the equivalence relation generated by collar maps and homeomorphisms isotopic to the identity \textbf{extended isotopy}.

\end{document}
