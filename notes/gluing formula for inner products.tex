\documentclass{gwiki}

\Title{Gluing formula for inner products}
\Tags{TQFT, unitary-TQFT, gluing}
\CreatedOn{2025-12-02 19:52}
\ModifiedOn{December 2, 2025 at 8:36 pm ET}

\begin{document}

\NoteHeader

\begin{lemma}[Gluing formula for inner products]
Let $(\mathcal{F}, \mathcal{U}, z)$ be an $(n{+}1)$-dimensional unitary TQFT. Suppose $X \in \mathcal{M}_n$ has boundary
\[
\partial X = (\overline{Y} \sqcup Y)_{\overline{\partial Y} \sqcup \partial Y} \cup Q
\]
for $Y, Q \in \mathcal{M}_{n-1}$. Fix $d \in \mathcal{F}(\partial Y)$, $q \in \mathcal{F}(Q; \overline{d} \sqcup d)$, and $y \in \mathcal{F}(Y, d)$. Then for $f, f' \in \mathcal{A}(X, (\overline{y} \sqcup y) \bullet q)$,
\[
\langle f_{\mathrm{gl}}, f'_{\mathrm{gl}} \rangle_{X_{\mathrm{gl}}, q_{\mathrm{gl}}}
= \sum_{s \in \mathrm{Irr}} \sum_{k=1}^{n_s}
\frac{\langle f \cup e_s^{(k)}, f' \cup e_s^{(k)} \rangle_{X, (\overline{y} \sqcup y) \bullet q}}{\langle e_s^{(k)}, e_s^{(k)} \rangle_{Y \times I, \overline{y} \sqcup y}},
\]
where $\mathrm{Irr} = \mathrm{Irr}(\mathsf{Rep}(\mathrm{End}_{\mathcal{A}(Y,d)}(y)))$, and $\{e_s^{(k)}\}$ is a complete set of orthogonal minimal idempotents for the unitary algebra $\mathrm{End}_{\mathcal{A}(Y,d)}(y)$.
\end{lemma}

\begin{proof}
\begin{itemize}
\item \textbf{Setup and notation.} The gluing $X_{\mathrm{gl}}$ is obtained by identifying the copy of $\overline{Y}$ in $\partial X$ with the copy of $Y$, yielding $\partial X_{\mathrm{gl}} = Q_{\mathrm{gl}}$. After gluing, the field $(\overline{y} \sqcup y) \bullet q$ on $\partial X$ descends to the glued field $q_{\mathrm{gl}}$ on $\partial X_{\mathrm{gl}}$. For any $n$-manifold $M$ with boundary conditions $c$, the inner product on $\mathcal{A}(M;c)$ is defined by the path integral of the cylinder (\wref[Walker06, 6.2.1]{WalkerTQFTnotes06})
\[
\langle x, y \rangle_{M,c} = Z(M \times I)(\overline{x} \cup y),
\]
where $\overline{x}$ is the conjugated field on $\overline{M} = M \times \{0\}$ and $y$ sits on $M \times \{1\}$. The cylinder category $\mathcal{A}(Y;d)$ has objects $\mathcal{C}(Y;d)$ and morphisms $\mathcal{A}(Y;d)^1_{ab} = \mathcal{A}(Y \times I; \overline{a}, b)$. Composition is given by stacking cylinders. In particular, the endomorphism space
\[
\mathrm{End}_{\mathcal{A}(Y,d)}(y) \coloneqq \mathcal{A}(Y \times I; \overline{y}, y)
\]
is an associative algebra under composition.

\item \textbf{Step 1. Applying the gluing formula to $X_{\mathrm{gl}} \times I$.} To compute $\langle f_{\mathrm{gl}}, f'_{\mathrm{gl}} \rangle_{X_{\mathrm{gl}}}$, we evaluate
\[
\langle f_{\mathrm{gl}}, f'_{\mathrm{gl}} \rangle = Z(X_{\mathrm{gl}} \times I)(\overline{f}_{\mathrm{gl}} \bullet f'_{\mathrm{gl}}).
\]
The manifold $X_{\mathrm{gl}} \times I$ is itself obtained by gluing $(X \times I)$ along the cylinder $Y \times I$ (since $(X \times I)_{\mathrm{gl}} = X_{\mathrm{gl}} \times I$). Applying the gluing formula \wref[Walker, (6.1.9)]{WalkerTQFTnotes06} for an orthogonal basis $\{e_i\}$ of the gluing locus $\mathcal{A}(Y \times I; \overline{y}, y)$,
\[
Z(X_{\mathrm{gl}} \times I)(\overline{f}_{\mathrm{gl}} \bullet f'_{\mathrm{gl}}) = \sum_i \frac{Z(X \times I)\bigl((\overline{f} \cup \overline{e}_i) \bullet (f' \cup e_i)\bigr)}{\langle e_i, e_i \rangle_{Y \times I}}.
\]
The numerator simplifies using compatibility of conjugation with $\cup$ via
\[
Z(X \times I)\bigl((\overline{f} \cup \overline{e}_i) \bullet (f' \cup e_i)\bigr) = Z(X \times I)\bigl(\overline{f \cup e_i} \bullet (f' \cup e_i)\bigr) = \langle f \cup e_i, f' \cup e_i \rangle_X.
\]
Thus
\[
\langle f_{\mathrm{gl}}, f'_{\mathrm{gl}} \rangle = \sum_i \frac{\langle f \cup e_i, f' \cup e_i \rangle_X}{\langle e_i, e_i \rangle_{Y \times I}},
\]
for any orthogonal basis $\{e_i\}$ of $\mathrm{End}_{\mathcal{A}(Y,d)}(y)$.

\item \textbf{Step 2. Structure of the endomorphism algebra.} The algebra $E \coloneqq \mathrm{End}_{\mathcal{A}(Y,d)}(y)$ is finite-dimensional. It carries an inner product from the cylinder $\langle e, e' \rangle = Z(Y \times I \times I)(\overline{e} \bullet e')$, making it a \emph{unitary algebra} (the inner product is compatible with the $*$-structure given by conjugation). By semisimplicity (a consequence of unitarity and finite-dimensionality), $E$ decomposes as a direct sum of matrix algebras
\[
E \cong \bigoplus_{s \in \mathrm{Irr}} M_{n_s}(\mathbb{C}),
\]
where $\mathrm{Irr}$ indexes the simple modules and $n_s = \dim(s)$. A unitary semisimple algebra admits a complete set of \emph{orthogonal minimal idempotents} $\{e_s^{(k)}\}$, for $s \in \mathrm{Irr}$ and $k = 1, \ldots, n_s$. These satisfy the multimatrix unit relations
\[
e_s^{(k)} \circ e_t^{(\ell)} = \delta_{st}\delta_{k\ell}\, e_s^{(k)}.
\]

\item \textbf{Step 3. The minimal idempotents form an orthogonal basis.} We verify that $\{e_s^{(k)}\}$ is orthogonal with respect to $\langle \cdot, \cdot \rangle_{Y \times I}$. The inner product on $E = \mathcal{A}(Y \times I; \overline{y}, y)$ satisfies the \emph{trace property} (Walker [6.2.4])
\[
\langle e \circ e', e'' \rangle = \langle e, e'' \circ (e')^* \rangle
\]
for composable morphisms, where $(-)^*$ is the conjugate-adjoint. For idempotents, $e^* = e$ (they are self-adjoint projections in a unitary algebra). Computing for two idempotents
\[
\langle e_s^{(k)}, e_t^{(\ell)} \rangle = \langle e_s^{(k)} \circ e_t^{(\ell)}, \mathrm{id}_y \rangle = \delta_{st}\delta_{k\ell}\, \langle e_s^{(k)}, \mathrm{id}_y \rangle.
\]
When $(s,k) \neq (t,\ell)$, the idempotent relation gives $e_s^{(k)} \circ e_t^{(\ell)} = 0$, so $\langle e_s^{(k)}, e_t^{(\ell)} \rangle = 0$. When $(s,k) = (t,\ell)$, positive-definiteness of the inner product gives $\langle e_s^{(k)}, e_s^{(k)} \rangle > 0$. Thus $\{e_s^{(k)}\}_{s,k}$ is indeed an orthogonal basis of $E$.

\item \textbf{Step 4. Conclusion.} Substituting the orthogonal basis $\{e_s^{(k)}\}$ into the formula from \textbf{Step 1},
\[
\langle f_{\mathrm{gl}}, f'_{\mathrm{gl}} \rangle_{X_{\mathrm{gl}}, q_{\mathrm{gl}}} = \sum_{s \in \mathrm{Irr}} \sum_{k=1}^{n_s} \frac{\langle f \cup e_s^{(k)}, f' \cup e_s^{(k)} \rangle_{X, (\overline{y} \sqcup y) \bullet q}}{\langle e_s^{(k)}, e_s^{(k)} \rangle_{Y \times I, \overline{y} \sqcup y}}.
\]
\end{itemize}
\end{proof}

\begin{remark}[Simple case]
When $y$ is \emph{simple}, meaning $\mathrm{End}(y) \cong \mathbb{C}$, there is a unique minimal idempotent $e = \mathrm{id}_y$, and the formula becomes
\[
\langle f_{\mathrm{gl}}, f'_{\mathrm{gl}} \rangle = \frac{\langle f, f' \rangle_X}{\langle \mathrm{id}_y, \mathrm{id}_y \rangle_{Y \times I}} = \frac{\langle f, f' \rangle_X}{d_y},
\]
where $d_y = \langle \mathrm{id}_y, \mathrm{id}_y \rangle$ is the quantum dimension.
\end{remark}

\SeeAlso{next}

\References

\Footer

\end{document}
