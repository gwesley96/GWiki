\documentclass{gwiki}

\Title{Functor}
\Tags{category-theory, definition}
\Summary{A functor is a structure-preserving map between categories, sending objects to objects and morphisms to morphisms while preserving composition and identities.}

\begin{document}

\NoteHeader

\begin{idea}[Idea]
Functors are to categories what homomorphisms are to groups: they preserve the categorical structure by mapping objects and morphisms in a way that respects composition and identities, enabling us to compare and relate different mathematical contexts.
\end{idea}

\begin{definition}[Functor]
Let $\cC, \cD$ be \wref[category]{category}. A \textbf{functor} $F : \cC \to \cD$ consists of:
\begin{lst}
  \item Object mapping: $A \mapsto F(A)$
  \item Morphism mapping: $(f : A \to B) \mapsto (F(f) : F(A) \to F(B))$
\end{lst}
preserving identities and composition:
\[
  F(\id_A) = \id_{F(A)}, \qquad F(g \circ f) = F(g) \circ F(f)
\]
\end{definition}

\section{Examples}

\begin{example}[Forgetful functor]
$U : \Grp \to \Set$ sends each group to its underlying set.
\end{example}

\begin{example}[Free functor]
$F : \Set \to \Grp$ sends $X$ to the free group on $X$.
\end{example}

\begin{example}[Hom functor]
For any $A \in \cC$: $\Hom(A, -) : \cC \to \Set$ is covariant.
\end{example}

\seealso{category, natural transformation}

\References

\Footer

\end{document}