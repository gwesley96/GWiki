\documentclass{gwiki}
\usepackage{gwiki-links}

\Title{tensor field}
\Tags{}

\begin{document}
\NoteHeader

#article 
\subsubsection*{Tensors }
Here we follow ((Lee18, pp. 393–5)). Let $V$ be a finite-dimensional $\mathbb{R}$-vector space, say of dimension $n$. A \textbf{covariant $\ell$-tensor} is a multilinear map $V^{\times \ell}\to \mathbb{R}$. A \textbf{contravariant $k$-tensor} is a multilinear map $(V^{*})^{\times k}\to\mathbb{R}$. 

A \textbf{$k$-contravariant, $\ell$-covariant tensor} (AKA \textbf{mixed $(k,\ell)$-tensor} or \textbf{$(k,\ell)$-tensor}) is a multilinear map $(V^*)^{\times k}\times V^{\ell}\to\mathbb{R}$, though we will also call multilinear maps whose arguments still consist of $k$ covectors and $\ell$ vectors, but not necessarily in the order given here; such an object is still called a tensor of type $(k,\ell)$. We write $\mathrm{T}^{\ell}(V)\coloneqq\set{\text{covariant \(\ell\)-tensors on \(V\)}}$ and $\mathrm{T}^{(k,\ell)}(V)\coloneqq\{(k,\ell)\text{-tensors on }V\}$. For $F\in \mathrm{T}^{(k,\ell)}(V)$ and $G\in\mathrm{T}^{(p,q)}(V)$, define $F\otimes G\in \mathrm{T}^{(k+p,\ell+q)}(V)$ by$$    (F\otimes G)(\omega^1,\dots,\omega^{k+p},v_1,\dots,v_{\ell+q})\coloneqq F(\omega^1,\dots,\omega^k,v_1,\dots,v_\ell)G(\omega^{k+1},\dots,\omega^{k+p},v_{\ell+1},\dots,v_{\ell+q}).$$If $e_i$ is a basis for $V$ and $\varepsilon^i$ is the associated \wref{dual basis} for $V^*$, then the set$$\set{e_{i_1}\otimes\cdots\otimes e_{i_k}\otimes \varepsilon^{j_1}\otimes\cdots\otimes\varepsilon^{j_\ell}}{i_q,j_q\in \{1,\dots,n\}}$$is a basis for $\mathrm{T}^{(k,\ell)}(V)$. As multilinear maps, these are given by$$(e_{i_1}\otimes\cdots\otimes e_{i_k}\otimes \varepsilon^{j_1}\otimes\cdots\otimes\varepsilon^{j_\ell})(\varepsilon^{s_1},\dots,\varepsilon^{s_k},e_{r_1},\dots,e_{r_{\ell}})=\delta^{s_1}_{i_1}\cdots\delta_{i_k}^{s_k}\delta_{r_1}^{j_1}\cdots\delta_{j_{\ell}}^{r_{\ell}}.$$Thus $\dim\mathrm{T}^{(k,\ell)}(V)=n^{k+\ell}$, and every $(k,\ell)$-tensor $F$ can be written as$$F=F_{j_1\cdots j_\ell}^{i_1\cdots i_k} \,e_{i_1}\otimes\cdots \otimes e_{i_{k}}\otimes \varepsilon^{j_1}\cdots \varepsilon^{j_\ell}$$where $F_{j_1\cdots j_\ell}^{i_1\cdots i_k}=F(\varepsilon^{i_1},\dots,\varepsilon^{i_{k}},e_{j_1},\dots,e_{j_{\ell}})$. 

\textbf{Example.} If the arguments of a $(k,\ell)$-tensor appear in a nonstandard order, then the horizontal positions of the indices in the component functions should reflect this. For example, if $A$ is a $(1,2)$-tensor with first argument a vector, second argument a covector, and third a vector, then its basis expression would be written$$A = A^i{}_j{}^k\, \varepsilon^i\otimes e_j\otimes \varepsilon^k$$where $A^i{}_j{}^k = A( e_i,\varepsilon^j,e_k)$.

The \textbf{contraction} of a $(k,\ell)$-tensor $F$ on the pair $(i_p,j_q)$ where $i_p$ is a covariant (lower) index and $j_q$ is a contravariant (upper) index is a $(k-1,\ell-1)$-tensor obtained by taking the trace on indices, i.e., by replacing both with a dummy index $k$ and summing over $k$. For example, if we contract the above $(3,1)$-tensor $A$ on its first and second indices, we get the covariant 1-tensor $B$ with components $B_k=A_i{}^i{}_k$.

\subsubsection*{Tensor fields}
A $(k,\ell)$-\textbf{tensor field} is a \wref{section} of the so-called \textbf{$(k,\ell)$-tensor bundle}, i.e., of the vector bundle $T^k M \otimes T^\ell M$ over $M$.

\textbf{Example.} A $(0,2)$-tensor field is a map $M \to T^*M \otimes T^*M$, written in local coordinates $(x^i)$ as $f_{ij} \, \mathrm{d} x^i \otimes \mathrm{d} x^j$ for some $f_{ij} \in C^\infty(M)$.


#unorganized 
, e.g., converting a covector (a 1-form, whose entries are indexed by subscripts) to a vector (a vector, whose entries are indexed by superscripts).

\end{document}
