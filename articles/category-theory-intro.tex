\documentclass[type=article,gwikiid=category-theory-intro]{gwiki}

%% ============================================================================
%% Introduction to Category Theory
%% GWiki Article - Example
%% ============================================================================

\GWikiMeta{category-theory-intro}{Introduction to Category Theory}{article}[category-theory, math, foundations]
\GWikiAuthor{GWiki}
\GWikiDate{2025-12-08}
\GWikiSummary{%
  A gentle introduction to category theory, covering the basic definitions
  of categories, functors, and natural transformations with motivating examples.
}

\begin{document}

\gwikiarticleheader

\tableofcontents

%% ============================================================================
\section{Introduction}
%% ============================================================================

Category theory provides a unifying language for mathematics, allowing us to see
common patterns across different mathematical structures. Rather than studying
objects in isolation, category theory emphasizes the relationships (morphisms)
between objects.

This article introduces the fundamental concepts. For the precise definition of
a category, see the wiki note \wref{category}[Category]. For functors, see
\wref{functor}[Functor].

%% ============================================================================
\section{Categories}
%% ============================================================================

\begin{definition}[Category]
  A \concept{category} $\cC$ consists of:
  \begin{lst}
    \item A collection of \term{objects}, denoted $\Obj(\cC)$
    \item For each pair of objects $A, B$, a collection of \term{morphisms}
          $\Hom_\cC(A, B)$
    \item For each object $A$, an \term{identity morphism} $\id_A : A \to A$
    \item A \term{composition operation}: given $f : A \to B$ and $g : B \to C$,
          we get $g \circ f : A \to C$
  \end{lst}
  satisfying associativity and unit laws.
\end{definition}

The key insight is that morphisms, not objects, carry the essential information.

\subsection{Examples.}

\begin{example}[The category $\Set$]
  The category $\Set$ has sets as objects and functions as morphisms.
  This is perhaps the most familiar category.
\end{example}

\begin{example}[The category $\Grp$]
  The category $\Grp$ has groups as objects and group homomorphisms as morphisms.
\end{example}

\begin{example}[The category $\Top$]
  The category $\Top$ has topological spaces as objects and continuous maps
  as morphisms.
\end{example}

%% ============================================================================
\section{Functors}
%% ============================================================================

Once we have categories, we can ask: what are the structure-preserving maps
between categories? These are \term{functors}.

\begin{definition}[Functor]
  A \concept{functor} $F : \cC \to \cD$ between categories consists of:
  \begin{lst}
    \item A mapping on objects: $A \mapsto F(A)$
    \item A mapping on morphisms: $(f : A \to B) \mapsto (F(f) : F(A) \to F(B))$
  \end{lst}
  preserving identities and composition.
\end{definition}

For more details, see the wiki note \wref{functor}.

\subsection{The Forgetful Functor.}

\begin{example}
  The \term{forgetful functor} $U : \Grp \to \Set$ sends each group to its
  underlying set, ``forgetting'' the group structure.
\end{example}

%% ============================================================================
\section{Natural Transformations}
%% ============================================================================

Having defined morphisms between categories (functors), we can ask: what are
morphisms between functors? These are \term{natural transformations}.

\begin{definition}[Natural Transformation]
  Given functors $F, G : \cC \to \cD$, a \concept{natural transformation}
  $\a : F \To G$ consists of morphisms $\a_A : F(A) \to G(A)$ for each object
  $A$ in $\cC$, such that for every morphism $f : A \to B$, the following
  diagram commutes:
  \[
    \begin{tikzcd}
      F(A) \ar[r, "\a_A"] \ar[d, "F(f)"'] & G(A) \ar[d, "G(f)"] \\
      F(B) \ar[r, "\a_B"'] & G(B)
    \end{tikzcd}
  \]
\end{definition}

For more details, see \wref{natural-transformation}.

%% ============================================================================
\section{The Category of Categories}
%% ============================================================================

An important observation is that categories, functors, and natural transformations
themselves form a structure:

\begin{itemize}
  \item Categories are objects
  \item Functors are morphisms
  \item Natural transformations are ``2-morphisms''
\end{itemize}

This leads to the notion of a \term{2-category}, which we explore in
\articleref{higher-categories}[the article on higher categories].

%% ============================================================================
\section{A Diagram Example}
%% ============================================================================

Here is a more complex TikZ diagram demonstrating the package capabilities:

\begin{center}
\begin{tikzpicture}[
  node distance=2cm,
  arrow/.style={->, thick},
  doublearrow/.style={double, ->, thick}
]
  % Objects
  \node (C) at (0,0) {$\cC$};
  \node (D) at (3,0) {$\cD$};
  \node (E) at (6,0) {$\cE$};

  % Functors
  \draw[arrow, bend left=30] (C) to node[above] {$F$} (D);
  \draw[arrow, bend right=30] (C) to node[below] {$G$} (D);
  \draw[arrow] (D) to node[above] {$H$} (E);

  % Natural transformation
  \draw[doublearrow] (1.5,0.5) to node[right] {$\a$} (1.5,-0.5);
\end{tikzpicture}
\end{center}

%% ============================================================================
\section{Conclusion}
%% ============================================================================

Category theory provides a powerful framework for understanding mathematical
structures through their relationships. The trinity of categories, functors,
and natural transformations forms the foundation for much of modern mathematics.

\seealso{category, functor, natural-transformation, yoneda-lemma}

\gwikifooter

\end{document}
