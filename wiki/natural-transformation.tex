\documentclass[type=wiki,gwikiid=natural-transformation]{gwiki}

%% ============================================================================
%% Natural Transformation
%% GWiki Note - Definition
%% ============================================================================

\GWikiMeta{natural-transformation}{Natural Transformation}{wiki}[category-theory, definition]
\GWikiDate{2025-12-08}
\GWikiSummary{%
  A natural transformation is a morphism between functors, consisting of
  component morphisms that satisfy a naturality condition.
}

\begin{document}

\gwikinoteheader

%% ============================================================================
%% Definition
%% ============================================================================

\begin{definition}[Natural Transformation]
  Let $F, G : \cC \to \cD$ be \wref{functor}[functors]. A \concept{natural transformation}
  $\a : F \To G$ consists of:
  \begin{lst}
    \item For each object $A \in \cC$, a morphism $\a_A : F(A) \to G(A)$ in $\cD$,
          called the \term{component} of $\a$ at $A$
  \end{lst}
  such that for every morphism $f : A \to B$ in $\cC$, the following
  \term{naturality square} commutes:
  \[
    \begin{tikzcd}
      F(A) \ar[r, "\a_A"] \ar[d, "F(f)"'] & G(A) \ar[d, "G(f)"] \\
      F(B) \ar[r, "\a_B"'] & G(B)
    \end{tikzcd}
  \]
  That is, $G(f) \circ \a_A = \a_B \circ F(f)$.
\end{definition}

%% ============================================================================
%% Special Cases
%% ============================================================================

\begin{definition}[Natural isomorphism]
  A natural transformation $\a : F \To G$ is a \concept{natural isomorphism}
  if each component $\a_A$ is an isomorphism. We write $F \cong G$.
\end{definition}

\begin{remark}
  Natural isomorphisms are the ``right'' notion of equivalence between functors.
\end{remark}

%% ============================================================================
%% Composition
%% ============================================================================

\begin{proposition}[Vertical composition]
  Given $\a : F \To G$ and $\b : G \To H$, define $\b \circ \a : F \To H$ by
  $(\b \circ \a)_A = \b_A \circ \a_A$. This is a natural transformation.
\end{proposition}

\begin{proposition}[Horizontal composition]
  Given $\a : F \To G$ with $F, G : \cC \to \cD$ and $\b : H \To K$ with
  $H, K : \cD \to \cE$, the \term{horizontal composite} $\b * \a : H \circ F \To K \circ G$
  is defined by $(\b * \a)_A = \b_{G(A)} \circ H(\a_A) = K(\a_A) \circ \b_{F(A)}$.
\end{proposition}

%% ============================================================================
%% Examples
%% ============================================================================

\begin{example}[Double dual]
  For finite-dimensional vector spaces, there is a natural isomorphism
  $\a : \id_{\fdVec} \To (-)^{**}$ given by $\a_V(v)(\f) = \f(v)$.
  This is ``canonical'' (no choice of basis required).
\end{example}

\begin{example}[Determinant]
  The determinant $\det : \GL_n \To (-)^*$ is a natural transformation
  from the general linear group functor to the units functor.
\end{example}

%% ============================================================================
%% Diagram
%% ============================================================================

The naturality condition as a diagram:

\begin{center}
\begin{tikzpicture}[scale=1.2]
  % Top row
  \node (FA) at (0,2) {$F(A)$};
  \node (GA) at (3,2) {$G(A)$};

  % Bottom row
  \node (FB) at (0,0) {$F(B)$};
  \node (GB) at (3,0) {$G(B)$};

  % Horizontal arrows
  \draw[->] (FA) -- node[above] {$\a_A$} (GA);
  \draw[->] (FB) -- node[below] {$\a_B$} (GB);

  % Vertical arrows
  \draw[->] (FA) -- node[left] {$F(f)$} (FB);
  \draw[->] (GA) -- node[right] {$G(f)$} (GB);
\end{tikzpicture}
\end{center}

%% ============================================================================
%% Related Concepts
%% ============================================================================

\seealso{functor, category, yoneda-lemma, adjunction}

\gwikifooter

\end{document}
