\documentclass[type=wiki,gwikiid=category]{gwiki}

%% ============================================================================
%% Category
%% GWiki Note - Definition
%% ============================================================================

\GWikiMeta{category}{Category}{wiki}[category-theory, definition, foundations]
\GWikiDate{2025-12-08}
\GWikiSummary{%
  A category is a mathematical structure consisting of objects, morphisms,
  and composition satisfying associativity and identity laws.
}

\begin{document}

\gwikinoteheader

%% ============================================================================
%% Definition
%% ============================================================================

\begin{definition}[Category]
  A \concept{category} $\cC$ consists of the following data:
  \begin{lst}
    \item A class $\Obj(\cC)$ of \term{objects}
    \item For each pair of objects $A, B \in \Obj(\cC)$, a set $\Hom_\cC(A, B)$
          of \term{morphisms} (or \term{arrows}) from $A$ to $B$
    \item For each object $A$, an \term{identity morphism} $\id_A \in \Hom_\cC(A, A)$
    \item For each triple of objects $A, B, C$, a \term{composition function}
          \[
            \circ : \Hom_\cC(B, C) \times \Hom_\cC(A, B) \to \Hom_\cC(A, C)
          \]
  \end{lst}
  satisfying:
  \begin{lst}
    \item \textbf{Associativity:} $(h \circ g) \circ f = h \circ (g \circ f)$
    \item \textbf{Identity:} $f \circ \id_A = f = \id_B \circ f$ for all $f : A \to B$
  \end{lst}
\end{definition}

%% ============================================================================
%% Notation
%% ============================================================================

\begin{notation}
  We write $f : A \to B$ or $A \xto{f}{} B$ to indicate $f \in \Hom_\cC(A, B)$.
  The collection $\Hom_\cC(A, B)$ is also denoted $\cC(A, B)$ or $\Mor_\cC(A, B)$.
\end{notation}

%% ============================================================================
%% Examples
%% ============================================================================

\begin{example}[Standard categories]
  \begin{lst}
    \item $\Set$: sets and functions
    \item $\Grp$: groups and homomorphisms
    \item $\Ab$: abelian groups and homomorphisms
    \item $\Ring$: rings and ring homomorphisms
    \item $\Top$: topological spaces and continuous maps
    \item $\Vec_k$: vector spaces over $k$ and linear maps
  \end{lst}
\end{example}

\begin{example}[Small categories]
  A \term{small category} is one where $\Obj(\cC)$ is a set (not a proper class).
  Any monoid $(M, \cdot, e)$ defines a small category with one object $*$ and
  $\Hom(*, *) = M$.
\end{example}

%% ============================================================================
%% Properties
%% ============================================================================

\begin{remark}[Size issues]
  The distinction between sets and proper classes is important. A category
  is \term{locally small} if each $\Hom_\cC(A,B)$ is a set.
\end{remark}

%% ============================================================================
%% Related Concepts
%% ============================================================================

\seealso{functor, natural-transformation, morphism}

%% ============================================================================
%% References
%% ============================================================================

For a comprehensive introduction, see \articleref{category-theory-intro}.

\gwikifooter

\end{document}
